
% Me First
% Ajahn Candasiri

This evening I would like to talk about the \textit{Brahma Vih\=aras}. These are states that arise quite naturally when the mind is free from self-interest. They are the lovely boundless qualities of \textit{\textit{mett\=a}} -- kindliness, \textit{\textit{karu\d{n}\=a}} -- compassion, \textit{mudit\=a} -- sympathetic joy, or gladness at the beauty, the good fortune of others, and \textit{upekkh\=a} -- equanimity, or serenity. I really like this teaching because even though, mostly, we are nowhere near that level of pure radiance, they are qualities that we can bring forth in smaller ways.

Over the years my understanding of them has altered quite radically. I used to try radiating kindness out to others, having compassion for others, delighting in the good fortune of others and finding equanimity in the midst of suffering -- but I could never really do it, it never seemed to work very well. When I was trying to be glad at the good fortune of others, all I could feel was jealous. When I was trying to feel equanimous about the suffering around me, all I could feel was disturbed and upset. When I tried to feel compassion, all I could feel was anxiety: `What can I do to make it all right?' When I was trying to feel kindness towards people I didn't like very much and it didn't work, I used to just hate myself. I became thoroughly confused, so I realized that I had to try a different approach.

I remember when I first told people that I was going to be a nun, one immediate response was, `Well, how selfish! Isn't that awfully self-centred?' My reply was, `Yes. It's completely self-centred  \ldots{}  but until I can understand my own suffering, my own difficulty, I'm not going to be able to help anybody else very much.' Although I wanted to help, I saw that my capacity for serving others was very limited and that, really, I had to begin here.

We can easily talk about world peace and about caring for others, but actually cultivating loving-kindness demands a lot. It demands a broadening of the heart and our view of the world. For example, I've noticed that I can be very picky about things. There are some things that I can feel boundless kindness and love for -- but only as long as they are agreeable, and behaving in the way that I want them to behave! Even with people we love dearly, if they say something that is upsetting, a bit jarring, that channel of limitless, boundless love can close immediately; not to mention with the people that we don't like, or who might have different views from our own.

So it does take a bit of reflection to begin to broaden that sphere of \textit{mett\=a}, loving kindness. We may manage to do it in an idealistic, intellectual way; we may find we can spread \textit{mett\=a} to people we don't know or that we don't have to associate with, but that is very different from doing it with those we live with all the time. Then, it's not always so easy -- much as we may want to. This can be a source of anguish: `I really want to like this person -- but they drive me nuts!' I'm sure you have all experienced this with certain people. You may even feel sorry for them -- but they still drive you nuts! You can end up feeling that you \textit{should} love them, but somehow you just can't. I have found Ajahn Sumedho's interpretation of \textit{mett\=a} very helpful; he would say: `Well, to expect to \textit{love} somebody is maybe asking too much -- but at least refrain from nurturing thoughts of negativity and ill-will towards them.' So, for me, the starting point for cultivating \textit{mett\=a} has been simply the recognition of its absence, or even the presence of its opposite. 

For many years I had a kind of subliminal negativity going on; quietly grumbling away, usually about myself: `You're not good enough. You've been meditating all these years, and still your mind wanders and you fall asleep. You're \textit{never} going to be any good.' -- those kinds of voices. Are they familiar \ldots{} just quietly there, mumbling away, undermining any sense of well-being? It took me a long time to recognize how much negativity I was harbouring in my heart.

Then there can be grumbling about other people: `Look at the way she sits!', `Good heavens, he eats an awful lot!', \mbox{`I really} don't like the routine of this retreat. Why do we have to get up so early?' You'll all have your own niggles. The important thing is not \textit{not} to have them, but to recognize them -- to actually allow ourselves to be fully conscious of this grumbling, negative mind; and then to be \textit{very} careful not to add to the negativity by being negative about it: `I never realized what a terribly negative person I am. I'm a hopeless case!' That's not very helpful. Instead, we can begin to take a kindly interest: `Well, that's interesting. Fancy thinking like that; I never realized how much that mattered to me,' rather than hating ourselves for having such thoughts. 

One thing I've discovered is that often the things I find hardest to accept in others are things that I actually do myself. It can be quite humbling, but incredibly helpful, to notice what others do or say that is upsetting. Then to ask inwardly, `Is that something that I do?' Sometimes it can be difficult and painful to acknowledge but, fortunately, it can be a private process -- we don't have to tell anyone else! Then, as we begin to soften and find that capacity for accepting ourselves -- including all the foolishness, the inadequacy, the shyness -- the heart expands, and we are able to extend acceptance and kindness towards a much greater range of people and situations. So this quality of \textit{mett\=a}, of kindliness, has to start with this being here. We don't need to manufacture it, it arises quite naturally as we cultivate more kindliness and acceptance of ourselves.

This may seem strange if we have always been told we should think of others before ourselves but I have found that, in fact, trying to do it the other way round never really works. I might be able to do and say the right kinds of things, but often there'd be some quite harsh underlying judgement. For example, before I was sick, I used to feel critical of people who couldn't work as hard as I could. I'd say to them, `Yes, do be careful; do rest if you need to,' but I'd be thinking, `You're just so feeble; if you were practising correctly you'd be able to do it!' It was only after experiencing a state where, after ten minutes of work I'd need to rest for half an hour, that I really knew what that was like. Only then was I able to feel genuine kindness towards those in difficulty, or physically limited in some way.

As monastics, we make a commitment to harmlessness. However, the way our training works is to allow us to see directly those energies that maybe aren't so harmless, and aren't so beautiful: the powerful lust, sensuality or rage -- they all come bubbling up. It can be rather alarming at first but now, having experienced those energies within my own heart, I can understand much better the state of the world and the things that happen in it. Of course, I can't approve of many of the things I hear about, but there is much less tendency to judge or to blame.

\textit{Karu\d{n}\=a} -- compassion -- is the second \textit{Brahma Vih\=ara}. Looking at the word, `compassion': `-passion means `to feel', and `com-' is `with', so it's a `feeling with', or entering into suffering. Now one response, when confronted with a situation of pain or difficulty, can be to distance oneself. It may come from fear -- a feeling of, `I'm glad that's not happening to me,' so we do and say the right kinds of things but, actually, there's a standing back and a sense of awkwardness about what we are feeling. This is what can happen. It reminds me again of when I broke my back. There was one person who was very uncomfortable about it; when we met I could sense that she was shocked, and that she didn't want to get too close. But for me, her response brought quite a lonely feeling; it didn't really hit the spot. So \textit{karu\d{n}\=a} implies a willingness to actually take on board the suffering of another, to enter into it with them. It's a much fuller kind of engagement and demands an attunement with one's own heart; having compassion for one's own fear or awkwardness around another person's situation. 

Sometimes we feel awkward because we don't know what to say. If someone is bereaved or terminally ill, what do you say to them? We might be afraid of saying the wrong thing. However, when we are willing to be with our own sense of discomfort with the situation -- to bear with our own pain or suffering in relation to it -- we begin to sense the possibility of responding in a spontaneous, natural way.

This state of being fully present with another person in their difficulty is something that I trust now -- much more than any \textit{ideas} about a compassionate response. It's not about giving advice, or sharing the story of our aunt who had the same difficulty, or anything other than the willingness to be with the discomfort of the situation, to be with our own struggle. When we are fully present with suffering we find a place of ease, of non-suffering, and somehow we just know what is needed. It may be that nothing is needed, other than to be there; or perhaps something needs to be said, and suddenly we find ourselves saying just the right thing; or there may be some practical assistance we can offer. But none of this can happen until we have fully acknowledged our own struggle with what's happening. It needn't take more than a micro-second. We simply begin with tuning in to our own suffering, attending to that -- from this arises the compassionate response to the other.

\textit{Mudit\=a} is the quality of sympathetic joy. This one has always interested me greatly -- mostly because it was something that I often seemed to lack. I used to suffer enormously from jealousy, and there seemed to be nothing I could do about it. It would just come, and the more I tried to disguise it the worse it would get. I could really spoil things for people, just through this horrible thing that used to happen when I had a sense that somebody was in some way more fortunate or better than I was. 

Considering the three characteristics of existence (impermanence, unsatisfactoriness, and lacking in any permanent self-hood) was what brought a glimmer of hope -- to realize that jealousy is impermanent. Before, it would feel very, very permanent -- as though I would have to do something extremely major to get rid of it, to make it go away. It also felt like a very personal flaw. So this teaching enabled me to recognize that this was just a passing condition -- and that I didn't have to identify with it. It came, and it went. Certainly, it was extremely unpleasant -- but when I could let go of the struggle for things to be otherwise, it actually didn't stay very long; it would come and it would go, and that would be it.

People used to tell me about their \textit{mudit\=a} practice. They'd say: `Well, if I see somebody who has something better than me, I just feel really glad they have it' -- but I hadn't quite got to that point, I must say -- I didn't have such generosity of heart. I realized that there was still something missing. Eventually, I realized that what was lacking was \textit{mudit\=a} for myself. I realized that it was no good trying to have \textit{mudit\=a} for somebody else if what they had was something that I really wanted -- and thought I hadn't got!

So I saw that, rather than lamenting my own lack (which is basically what jealousy is, and which I would find so painful to acknowledge) I had to begin to look at what I had, to count my own blessings. This may sound a little strange in a context where many of us have been encouraged to rejoice in the goodness and beauty of others, but where the last thing we are supposed to do is to count our own blessings, or to think of how good we are ... but I saw that that was actually what I needed to do.

I'm sure that everyone here can find some things to feel glad about. Even if there are not very many things we can make much of the few, rather than pushing them to one side, saying: `No, they don't really count, that's nothing really -- but look at all these terrible faults I have!' We are very good at doing that -- but how good are we at looking at the goodness, the beauty of our lives? Everyone here can count the fact that they've chosen to come to the monastery, that there is a sincere interest in cultivating peace, as something to feel very glad about -- particularly seeing how many people are living their lives. We can also make much of the things we do well. Instead of, `Oh no, that wasn't very good,' we can try saying, `Well, actually, that was rather a beautiful thing that I did. I did do that well.' 

One person I know keeps a special diary. Whenever he does something good, he notes it in his diary. Then, when he is feeling a bit miserable, he reads it through -- then he feels much better! That struck me as really skilful -- a way of making much of goodness. Why not? Generally, we make so much of our misery and our inadequacy, why not instead try making much of the goodness of our lives? I began practising with this some time ago, and the more I've done it, the more naturally and spontaneously I can really feel happy when I hear of somebody else's success. Interesting, isn't it, how it works? So this is something I encourage you to contemplate: filling the heart with a sense of the beauty and goodness of \textit{your} life, as well as that of others. Then, when people are having a really joyful time together, instead of sort of sneering and looking down on them -- and feeling a bit lonely and isolated -- we can join in with appreciation, sharing in their delight. That is \textit{mudit\=a}.

\textit{Upekkh\=a}, equanimity, is the fourth \textit{Brahma Vih\=ara}. To me, it seems significant that the Buddha made so much of this quality. Sometimes this is translated as `indifference'; it's considered to be one of the highest spiritual states. Personally, I can't imagine that he would make much of `indifference', the way we often use this word -- a kind of not wanting to be bothered. So it's helpful to consider what this quality of equanimity, or serenity, actually feels like. What does it involve, in terms of the heart? One of the ways that I reflect now on \textit{upekkh\=a} is to see it as a grandness of heart whereby there is the willingness to touch and be in contact with it all -- from the most delightful state, to the most utterly wretched state. Having taken this human form, we are subject to wonderful, sublime experiences and, equally, to horrible, hellish miserable states. We can experience every realm of existence.

It's said that when the Buddha was dying, along with human beings, there were a great many \textit{devatas} (heavenly beings) gathered around as well. Those who weren't enlightened were tearing their hair, weeping, in a state of great anguish: `Oh no, he's dying, he's leaving us.' Whereas, the enlightened ones maintained perfect equanimity. They simply said: `It's in the nature of things to arise and to cease.' They acknowledged that things were happening according to their nature: having been born, things die. I think we've still got a bit of work to do there!

\textit{Upekkh\=a} implies an ability to stay steady amid the highs and lows of our existence -- amid the eight worldly winds: praise and blame, fame and insignificance, happiness and suffering, gain and loss -- these things that can affect us so much. When we contemplate life we see that sometimes things are good, and we feel great -- sometimes they're horrible. This is normal; we all have horrible times. Knowing that things change deepens our capacity to stay steady with them, instead of being completely thrown when they don't work out so well.

In our community (as in any family) there are times when it's wonderful; we all get along really well together. This is just fine, we can enjoy those times; but there are also days when it's horrible. In an international community it's easy for people to misunderstand each other, and sometimes there can be a tiff. Having experienced a great number of difficult times over the years, I'm now usually able to stay much more steady. It's still unpleasant, but there isn't such a strong reaction -- I don't feel that anything has gone terribly wrong. I've noticed too that somehow that capacity to stay steady helps everybody else to not be so badly thrown by what's happening. Whereas, when everybody is thrown, the whole thing escalates -- then, it really does get out of control!

Reflecting on the law of kamma, we see that everything happens as the result of what has gone before. So instead of blaming ourselves, or looking around for someone else to blame when things seem to go wrong, we can simply ask, `What is the lesson here? How can I work with this?' We can reflect that this is the result of kamma, and determine to maintain mindfulness; holding steady, rather than making the situation worse by some unskilful reaction. In this way we bring balance, not just into our own lives but for everyone else too. It's not such an easy thing but again, it comes back to acknowledging and making peace with our own sense of agitation, our own lack of equanimity.

Sometimes we come into contact with suffering that seems almost unbearable. When this happens I've noticed a tendency to try to protect myself, to try to shut out such impressions. But \textit{upekkh\=a} implies a willingness to expand the heart and to listen, even when things seem unbearable. As we practise, what seems to happen is that -- far from becoming indifferent -- we become able to encompass an ever-increasing range of experience; to be touched by life, as by a rich tapestry of infinite colour and texture. So it's definitely not a dumbing down or deadening, but rather an attunement to the totality of this human predicament. At the same time, we find an increasing capacity to hold steady with it: the balance of \textit{upekkh\=a}. Through establishing ourselves in the present, and holding steady -- whatever we might be experiencing -- we find the resources for dealing with \textit{whatever} we may encounter in this adventure we call `life'. 

Eva\d{m}.

