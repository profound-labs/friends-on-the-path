
% A question of balance
% Ajahn Candasiri

Every winter at our monasteries two or three months are set aside as quiet retreat time -- a time to focus more intensively on our inner work. The encouragement given during this time is towards cultivating a stiller, quieter space within the heart. For it is only through attention to this that we are able to observe all our skilful and less skilful habits, and to train the mind -- making it into a good friend, a good servant, rather than an enemy that can lead us into all kinds of unhappiness.

Emerging from such a period of retreat highlights a dilemma faced by many people -- whether living as householders, or in a monastic environment. The question it poses regards finding an appropriate balance between essential `inner work' -- which requires periods of withdrawal and seclusion -- and our relationship with `the world', including the responsibilities we have within our respective communities (whether family, or monastic Sa\.ngha) and towards the greater whole. If our attention and energies are directed only outwards towards our spiritual companions or towards society, it becomes clear sooner or later that even if we expend every ounce of energy right up until the last breath, there will still be more to do. The needs, the suffering of the world `out there' is endless. We can never make it \textit{all} all right. If we try, as many of us have to do before the penny finally drops, the result can be exhaustion, despair and disillusionment. 

Eventually we see that actually it's a question of balance. We need to find a way of balancing our `inner' work and our `outer' work -- and we begin to appreciate a basic paradox: that in order to be truly generous, truly of service to others, we actually need to be completely `self-centred'. We need to be able to stay in touch with our own hearts, listening carefully to what they tell us, even while engaged in external activity or interaction. We need to remain attentive to our own needs, and to really make sure that these are well taken care of -- even if it means disappointing people, letting them down, not living up to the expectations they may have of us (or that we have of ourselves). This is not at all easy, with the conditioning most of us have: `Don't be selfish'. 

There is a simile given by the Buddha of two acrobats. The master said to his pupil, `You watch out for me, and I will watch out for you. That way we'll show off our skill successfully and receive our reward.' But the pupil contradicted him, `But that won't do at all, Master. You should look after yourself, and I'll look after myself \ldots{} \textit{that} is how we shall perform successfully!' The Buddha then goes on to explain that, in a sense, it was the pupil who had got it right. Watching out for ourselves through the practice of mindfulness -- really applying ourselves to cultivating of the Four Foundations of Mindfulness -- benefits others \ldots{} and being mindful in regard to others is a way of taking care of our own hearts. Furthermore, we care for others through patience, gentleness and kindly consideration; this also is a way of protecting our own hearts.

During his lifetime, the Buddha established the Fourfold Assembly\footnote{\textit{Fourfold Assembly}: The community of monks, nuns, laymen and laywomen.} as a social structure that would facilitate the cultivation and maintenance of the qualities of mindfulness and consideration of others. However -- whether we go forth as monks or nuns or live as householders -- one thing is clear: it's likely to take time. This practice has to be developed and worked at over a lifetime.

Usually, things don't just change and fall into place with the first glimmer of insight. We need to do the work of laying the foundation, using the tools and guidance that the Buddha presented. Even though these were presented over 2,500 years ago they still work well, having been used over generations by men and women to shape their lives -- to enable the ripening of the potential that each one of us has. It waits quietly in the heart for us to choose to make its cultivation the priority of our lives.
