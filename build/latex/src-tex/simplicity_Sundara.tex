
% Simplicity
% Ajahn Sundara

The time I spent in Thailand practising meditation quite intensively was a great learning experience. For nearly two and a half years it gave me the chance to be part of a culture that was incredibly different from ours in its outlook on life; it also provided an environment that made me realize how much my mind was conditioned by Western values, assumptions, prejudices and conceit.

At first, many aspects of that culture were totally alien to me. There were so many things that I found impossible to understand; but by the time I left, I felt quite at home. So I would like to share with you some facets of the time spent in this beautiful country.

In the rural area where the monastery is located the people are mostly farmers, simple people, living uncomplicated lives. Unlike us, they do not seem to be burdened with a lot of psychological concerns or existential crises. Their lives revolve around immediate needs such as food, sleep, getting through the day, and simple pleasures of life. Thai people are great at enjoying themselves!

When I first met my teacher, Ajahn Anan, he asked me how my practice was going. I said that one of my interests in coming to Thailand was to have the opportunity to continue to develop it. Then he asked me if I had any difficulties, so I explained to him how I had been practising and how I was feeling about it at the time. It was quite extraordinary, as I was talking, suddenly to sense that I had a strong mirror in front of me, and to see this `Me', going through its usual programme with clever justifications, suddenly turn into a big cloud of proliferation! This was a wonderful insight. With anybody else I might have been offended or felt that I had not been taken seriously but somehow, with him -- maybe because he was just himself and deeply at ease -- there was a huge sense of relief.

The way Thai people approach the teaching and themselves is deeply influenced by the Buddhist teaching and its psychology. Even their everyday language is mixed with many P\=a\d{l}i words. I remember noticing how their way of speaking about the mind/heart could seem quite cold-hearted to us. If you were going through some great suffering, some fear or painful memories, the teacher could just say: `Well, it's just \textit{kilesa} (unwholesome mental states).' Or, `Your heart is not happy?'

Strangely enough, in that context such statements would completely deflate the habit to think in terms of `Me' having a huge problem that needs to be sorted out. And there was always a strong and compassionate mirror and reflection. If anyone else had reduced my `problems' to a simple feeling of unhappiness I would have been really annoyed and felt dismissed but, with Ajahn Anan, in whom I had a deep trust, I was able to see the way my mind worked and, when there was confusion, to drop it. I would be reminded of the present moment by the question: `What's going on? Is your heart unhappy?' Of course, in the immediacy of the moment nothing was going on because the monastery, located in a lovely forest on the side of a mountain, had a very peaceful atmosphere. It was a simple place, quiet and secluded, and there was nothing to do all day except receive alms food, eat and sweep one's path for about half an hour. That was all. The rest of the time was free to develop formal practice. My mind calmed down a lot.

These experiences gave me a real taste for simplicity, and for the mind in its state of normality -- the mind that does not create problems out of the way things are. I'm not saying that this seemingly simple and direct approach to the mind is right or wrong, but I found that practising in this environment and culture over a period of two years had a powerful effect. It helped me to stop the habit of creating myself as a person -- and this was quite a liberating thing. As the mind calmed down I could see the person, the sense of `Self', really clearly every time it arose.

\looseness=1
The teaching pointed out that if we suffer through the sense of Self, we can't actually go very far in our practice; insight cannot arise deeply enough to cut off attachment. The whole culture facilitates this approach. If one thinks too much, people consider that one is on the verge of madness. Ask any Thai: when someone thinks too much, they'll say that she or he has a `hot heart' -- and if you are `hot' (\textit{`ron'}, in Thai), you're seen as deluded. To have \textit{`ronchai'} (hot heart) is quite negative, even insulting. People there are not much into thinking -- I am not saying that it is good or bad but they don't trust the thinking mind. This was very different from the culture I came from where thinking is worshipped, tons of books are written, and people rely on and trust the intellect a lot. So it was interesting to be in a culture that functioned so differently -- so much more intuitive, \mbox{more feminine.}

What struck me most when I came back to Europe was the complexity of the Western way of life, and I could see that having access to many traditions and teachers had turned our society on the spiritual level into a vast supermarket. This is not all negative, but it's extremely challenging for the mind that is already struggling with all it receives through the senses. No wonder people become neurotic after being exposed to so much information and so many choices!

When you're in the forest out there, you're just with a few birds, a few creepy-crawlies and nature all around. Days come, rise, and pass away with nothing much happening, and you get used to a very simple, peaceful rhythm. I found that extremely pleasant, and I knew that it was conducive to deepening the practice. In fact, I felt quite at home and very privileged to have that opportunity. The culture itself being predominantly Buddhist keeps things simple, the whole atmosphere was not one in which you felt intellectually stimulated; it's quite amazing the effect that this has on the mind. It would naturally slow down a lot, and become quite still. So I was scared to come back to the West and whenever I thought of coming back, my mind would conjure up the image of drowning in a huge ocean of thought -- not a terribly auspicious sign!

Even though I had to adapt and to follow the Thai maechee etiquette, `the Thai nuns' choreography', as I used to call it: walking in line to receive my food behind very young boys, crouching down every time I spoke to a monk -- it was little, compared to the blessings and support I received there.

So I was not sure that I would cope with the life and rhythm of Amaravati. I decided that if I had to teach, I would keep things simple; I would just speak about practice, just facts: \textit{\=an\=ap\=anasati}, the five khandhas, or Dependent Origination. I would not complicate people's lives with more words, concepts and ideas.

But it was a great lesson in letting go when, a few weeks ago, I went to teach at a Buddhist group. As I was being driven there, I said innocently to the leader of the group: `How do you see the weekend?' Of course, I already had some ideas: `I'll just meditate with them. I'll really teach them how to do it, rather than to think about it -- then we can share our experience afterwards.' But the person answered: `Well, Sister, we really want to talk with you about practice, and ask you questions, and have some discussion on Dhamma, and \ldots{}' I thought, `Oh dear! Never mind.' I just had to let go. I was reminded of the teaching of Luang Por Sumedho: `Just receive life as it is. Don't make a problem about it. Open yourself to the way things are.'

Thai people do not seem to suffer much from self-hatred; they don't even seem to know what it means! Once, out of curiosity, I asked an educated Thai woman who had come to visit me: `Do you ever dislike yourself?' And she said, `No, never.' I was amazed. She had just been talking to me about some very painful issues in her life yet she was not critical. Self-negativity does not seem to be part of their psychological make up, whereas we are riddled with it. So we have a difficult beginning, because the first step on this path is to have peace in one's heart -- which doesn't happen if there is a lot of self-hatred.

Fortunately Ajahn Sumedho, who is well acquainted with the Western mind, has devised a very good way of dealing with its tendency to dwell on the negative side of things and be critical -- just recognizing it, and receiving it within a peaceful space of acceptance, love and ease. This is a mature step as most of us find it very difficult to create a space around experience, we tend to absorb into what comes through our minds and to create a Self around it.

Let's say we experience boredom; if there is no mindfulness, we easily absorb into that feeling and become somebody who is bored, somebody who has got a problem with boredom and needs to fix it. This approach hugely complicates a simple experience like boredom. So all we need to do is to allow space inwardly to contemplate that feeling, instead of fixing it as a problem. In Thailand where psychology and the Buddhist teaching are so intertwined, Ajahn Anan would simply say: `Well it's just one of the hindrances.' Simple, isn't it? But often for us it can't be just an ordinary boredom -- it's got to be a very personal and special one!

One of the things that attracted me most to the Buddhist teaching was the simplicity of its approach -- I think this is what all of us would like to nurture in our practice and in our lives. The Buddha said: `Just look at yourself. Who are you? What do you think you are? Take a look at your eyes and at visual objects. How do you receive that experience of sense contact? What are the eyes, nose, tongue, body and ears? What are thoughts?' He asks us to inquire into sensory experience, rather than absorb into it and react to pain and pleasure. He said to just observe and actually see the nature of experience, very simply, directly, without fuss. Just bring peacefulness and calm into your heart, and take a look.

The sensory experience is really what creates our world. Without knowing its source and its effect, it's very difficult to get out of the vicious circle of `Me' and `my problem' that needs to be sorted out, or `Me, who loves it' -- this kind of push and pull agitates the heart further. So instead of pushing and pulling we take a good look and, without getting involved, we know things as they are: impermanent, unsatisfactory, not self. But this requires certain conditions, it doesn't happen by itself.

The first condition is peace and calm; without that, it's very difficult to see anything. That's why a lot of our practice is actually to bring the heart to a state of balance and calm. Most people are in a constant state of reactivity. If you ask them if they suffer, they say, `No'. They think they are perfectly fine. But someone who has seen the suffering of reactivity gradually comes to realize that it's not the best way to relate to life; it is very limited -- always the experience of `Self', `Me' and `You'. But as the sense of `Self' decreases, the reactivity decreases, too.

It is not so much the sense of Self that is the obstacle, it's our identification with it. The Four Noble Truths point to that very experience: the suffering of attachment to Self, the belief that one has a permanent ego. One teacher gave the example of Self being like a necklace; when the beads are held together by a thread it becomes a necklace, but as soon as the thread is cut, the whole thing falls apart.

I've spent many years observing closely the experience of the sense of Self. I remember in the early days when I was upset, Ajahn Sumedho would say to me: `Well you don't need to suffer about that. You've got the Refuges and \ldots{}' -- but this used to enrage me: `But what about Me? I'm suffering right now!' I felt he was just dismissing the huge personal problem, and wasn't taking me seriously. So for years I cherished that sense of Self without knowing it. I didn't think I was deluded, no, I was simply taking myself seriously!

In Thailand if you suffer and talk about it, you quickly get this funny sense that your practice has gone down the drain. This may be because in the quiet and simple life of a forest monastery, the formal practice is strongly rooted in the development of concentration, \textit{sam\=adhi}. It's a different approach there.

At Amaravati, the foundation of our practice is the Four Noble Truths which point again and again to suffering, its cause, its relinquishment and the path. It's not so easy here to get refined states of mind because we are constantly impinged upon by sense contacts: things, work, many strong-minded people living together, etc. Ajahn Sumedho teaches that to free the mind one just needs to put this teaching into practice right until the day we die.

I was struck by how soft and gentle the psyche of the Thai people is compared to ours; I found them generally very easy going. They like to laugh a lot, and basically life is no problem; if you make one you are considered kind of stupid. Even very simple villagers will think that if you make a problem out of life you are stupid. This was a nice contrast to our tendency to complicate life, and create problems around most things -- mainly because we have not been taught a better way. Our whole culture is based on the view that the world is understood through the thinking apparatus, rather than through the silent knowing, the awakened mind.

For our practice to bear fruit it's important to not make too much of ourselves. Basically, as long as we are fascinated by ourselves we will be bound to suffering. When the mind is undermined by streams of self-centred thoughts such as: `I don't like myself', `I think I've got a problem', and so on, it ends up being fed the wrong food, filled with unskilful states (\textit{akusala dhamma}). The realization of Dhamma is dependent also on the strength of our mind, so how can that come about if there is not a certain degree of positive energy? That's why \textit{mett\=a}, kindness and acceptance, is very important. We don't get a bright mind by filling it with a lot of negative states; that weakens it. Whether it's anger, greed, jealousy or despair, if their true nature is not seen they weaken the \textit{citta}, the heart. But when we see them in the light of mindfulness, then they have no power over us. Try meditating filling your heart with \textit{mett\=a}, then with miseries and then with joy -- you'll see the difference, it's quite simple. You can do the same with anger; bring up things that make you angry for a moment and see how it affects the heart. These are just conditioned mental states, but often we are not really aware of how they affect us; this is the work of delusion.

So knowing very clearly the difference between what is skilful and unskilful, not from an intellectual point of view but from wisdom, is great progress. The Buddha's teaching is like a map that helps us to recognize skilful and unskilful dhammas which we should learn to recognise and let go.

Remember, your heart is like a container filled with things that come from the past. If we've been a thief, or lazy or arrogant, or loving and generous in the past then we'll have certain habits. When we meditate, we receive the result of our habits -- we can't just throw them away at will. Wouldn't it be nice if we could? We'd all have been enlightened a long time ago! So bearing patiently and compassionately with one's kamma is very important.

One thing that has become clearer from my experience in Thailand is that while the actual practice is always here and now, it's also a gradual process, like developing a skill. It needs concentration, mindfulness and effort. They are the tools needed to gain insight into our attachments and to let them go. We're all here to liberate the heart from delusion, to learn how to live free from remorse or confusion. For the fruits of practice to arise in the heart, we need to develop these qualities of mind.

Here in the West we make a big deal about the body, and demand a lot from it. It's got to be healthy, strong and comfortable; whereas, in the East, it's made much less of. It's important of course, as without it we would not be able to practise but if it cracks up or deteriorates, there is no need to agitate the mind. So as meditators, if we talk too much about our body, or want to sleep a little bit more, we are basically considered a lousy practitioner! From a Buddhist perspective, it is the mind that is more important, since it will condition what happens when you die. When the mind is strong and healthy, then the body calms down naturally, and benefits a lot more than when we allow ourselves to be overwhelmed by concern for its well-being. This outlook gave me a more balanced perspective on the physical body, and a more detached way to deal with it. The mind can easily dwell upon negative aspects of oneself or other people. This is the easiest way of looking at life; the hardest thing is to actually train the heart to follow the path of goodness, \textit{kusala dhamma}, skilful dhamma. We may feel down or depressed but consider, it's only one mental state, one moment; do we want to perpetuate such a state for a lifetime? Or can we actually, through wisdom, realize that it's only one moment, one feeling, one thought? Such realization brings a real sense of urgency. If we are going to have feelings, or to think -- which we can't avoid -- we might as well guide our mind towards things that are skilful. Anything else just drags us down to hell, but actually we do this to ourselves quite a lot unknowingly.

So we have an option: we can stay in hell, the realm of miseries, or in heaven, the realm of happiness, or we can stay in a state of peace that comes from wisdom -- knowing that when pleasant feelings, pleasant experiences are present, that's heaven and when it's unpleasant, that's hell. The moment you know both as they are, that's freedom, isn't it? The mind doesn't linger -- that's the middle way. We can't control life and it takes time to go beyond wanting heaven or fearing hell. Just the way people walk or open doors, the way they speak or eat can send you to hell or heaven. It doesn't take much. Isn't that ridiculous? Sometimes we feel blissed out or we feel friendly to the whole universe, then, coming back to our dwelling place we find somebody's making a bit of noise, and suddenly we feel enraged. It doesn't take much, does it? So life is very unstable. Yet there is the knowing, that moment of freedom when you know: `Ah, this is a feeling, sense contact, ear, nose \ldots{}\thinspace'

The teaching of the Buddha, remember, is to know sense contact, its object and the effect it has on the heart. So we hear our neighbours making a lot of noise: `I'm going to tell them off, I can't stand it!' But when we are able to let go of it, we notice that we don't mind really \ldots{} but then the noise starts again, and finally we find ourselves in front of their door knocking: `Can you stop it!' And of course, if there is no wisdom at that moment, no mindfulness, then later on, we feel remorseful: `I feel awful, I shouldn't have done that.' and the whole cycle of suffering starts again.

The path shown by the Buddha is very simple. We need to remind ourselves again and again to have \textit{sati}, mindfulness. It's like an endless refrain: \textit{sati}, \textit{sati}, \textit{sati}. Where are we now? The practice of awareness is always in the present moment. There's no knowing in the future or the past. We can know a thought that takes us into the past or the future, but in the moment there is just knowing, awareness.

We can remind ourselves that we are all here to train and keep the practice simple. To know what nurtures the heart: truth, peace, calm, compassion, \textit{mett\=a}. When we have \textit{mett\=a}, the ego, the Self, can dissolve. Notice how when people have \textit{mett\=a} for us, peace arises in our heart, doesn't it? When people feel love towards us, we feel more peaceful and more calm. This is what we can do for ourselves too, and if all of us do this with one another, it will be a good basis for the practice.

I just want to leave you with this: let's keep things very simple and remind ourselves that whatever complicates life, it's not something to trust. It's more likely to be the work of my friend M\=ara, the Self or ego. When the heart is at peace and there is understanding, then things are quite cool, quite peaceful, quite okay. So I wish you to cultivate kindness and infinite patience towards yourself, and towards whatever resultant kamma you have to work through and that may be bothering you at this time. This is why the Buddha said that patience and endurance are the highest disciplines.

