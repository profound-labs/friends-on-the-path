
% Fearless Presence
% Ajahn Sundara 

The Buddhist path is a path of training the heart and mind: educating the heart to recognize Dhamma when it is present and when it's not. It is a path that leads to a greater sense of happiness, well-being, and confidence. It deepens our human experience and its gracefulness. 

And yet, many of us practise for years and still find ourselves frustrated, feeling that we aren't going anywhere, not getting the results that we expect. We are not happy. We wonder why a path that is supposed to lead us to happiness and peace does not seem to do the job. 

We often have expectations about the way things are and the way things should be. We have expectations about life, practice, and the results that we should get. How many of us are really willing to simply be aware and present with the mind, here and now, and see it clearly as it is? Isn't that what the teaching is constantly pointing to? The cultivation of awareness, the development of wisdom, and the Four Foundations of Mindfulness are all teachings pointing to being in the here and now, and from this perspective, clearly seeing the nature and content of the mind. 

The training of the heart is about awareness of the simplicity of the present moment. It is refreshing and freeing to be awake and to be content where we are and with what we have. Yet, while a part of us loves to be at peace, another loves excitement. One part is really happy to let go of the ego mechanisms, while another would rather die than let go. A part of us loves the idea of letting go when things are difficult, but another part can be quite terrified of liberation.

The Buddha foresaw this difficulty and put suffering at the centre of his teaching. When we meditate and are mindful of suffering, we can see its impermanence and feel motivated to free ourselves from it. The Buddha's teaching is for those who have seen the suffering of delusion. Delusion is a misperception of reality. Do we really experience ourselves as we are? Or do we experience ourselves as we wish to be?

We are reminded to be present, to be here and now but we easily get bored with that. We think, `Well, I have been in the here and now for years, and nothing seems to have changed much.' We mistake the path of practice with the path of making our life more interesting or becoming a happier person. We don't realize that this path is a path of liberation from ignorance. It can take us down avenues that are not necessarily comfortable. The journey of transformation and liberation is not a comfortable one -- it goes against the grain, it disturbs the status quo. Are we ready for that? Or are we just satisfied with being a little happier in relationships: with our mother, our partner, our job, and our lifestyle. Do we simply hope to make the conditioned world a bit more comfortable? 

\looseness=1
It is not wrong to wish for better conditions of life, because all sentient beings wish for happiness. The Buddha never said that happiness was wrong. But his teaching points to a type of happiness that is unshakeable, one that is not dependent on neurotic and obsessive desires. It is the happiness that comes from the realization of the end of suffering, and that leads us to a deeper confidence and true freedom. This means that no matter what happens to you in life, this happiness cannot be taken away from you. Isn't it wonderful? You can be old, sick, you may not be getting what you want, you may be in pain, you may be criticized, treated unfairly and yet, instead of being miserable, your mind remains happy. That is what the Buddha points to and the way that he teaches. 

Things can change but our heart ceases being dependent for its happiness on anything external, on any worldly things, or on what people think about us. People can think that we are stupid, but we know ourselves well enough that it's okay for them to think what they want to think. It's okay for your mother to see you as a failure, or for your partner to think that you are no good. People have the right to think the way they want and you have the same right. We don't need to agree or feel victimized, undermined or crushed with what the world is doing to us. This is what the Buddha's teaching is aimed at, an inner independence of the heart which brings true compassion. 

Are we ready for this path of awakening? Are we ready for this journey? Or are we still waiting for some kind of saviour to take us by the hand and tell us, `Okay my dear, everything is going to be fine. You'll be able to smell the flowers from the beginning of the path till the end. It will be lovely -- beautiful sky, lovely mountains, everybody will love you, your mind will be blissful and peaceful, and I promise you that it will be like that forever.' 

Well, the truth is that if you want to really live in the present moment, you have to be fierce. And what I mean by fierce is that sometimes what you are aware of may be sweet, but actually witnessing the delusion of the mind is not particularly sweet.

M\=ara is the personification of that which is called evil in Buddhism. M\=ara's presence is found throughout the Buddha's life story and his teachings. M\=ara is that which distracts us from the present moment, robbing us of clarity, insight, peace and sanity. Yet, M\=ara is nothing more than our deluded self! A Sufi teacher once said, `If you want to know where the devil is, just look at yourself.' No need to go very far -- if you want to find M\=ara just check out your comfort zones. 

So, are we ready to walk this path? Are we ready to be awake? 

A great chunk of our practice is about building a strong container that can withstand the power of our difficult experiences -- from this vulnerable, open, awakened state, building a refuge in which we can abide comfortably. The first step is what the Buddha calls \textit{s\={\i}la}, or ethics. Many of you know the Five Precepts of not harming, not stealing, not taking anything that doesn't belong to you, not abusing anybody else or yourself with your own sexuality or sensuality, not speaking in ways that are harmful, and not taking drugs or intoxicants that confuse your consciousness. Those precepts can be just formulas that we repeat by rote every week, or they can be pointers for reflection to build a more peaceful abiding in ourselves. Why? Because you may have noticed that every time we break any ethical standards, we experience a lot of regret and unhappiness. 

Why is it not easy to enter this path? Because sometimes it's like going into a battle -- waging a war between our worldly mind and the awakening heart, the Dhamma heart. It is not necessary to create a duality between the world and the heart, but the actual experience of seeing both levels is often felt like that -- conflicting energy, opposed desires. And this is what I mean when I talk about going into battle. We don't have to go into battle all the time, but on the path we have to face many desires that are not in line with \textit{s\={\i}la} or what we truly, deeply wish for ourselves.

How much room do we give to the Dhamma that is blossoming and strengthening in our own being, and how much room do we give to our ignorance and self-concerned desires? We want to be kind, loving, caring and patient but many of our habits are contrary to this. We find ourselves struggling, full of aversion, impatient, demanding, and critical. 

This path does not require that we surrender ourselves to some kind of higher entity who is going to work magic tricks on us. This path does not ask you to call upon some kind of divine intervention. It is a path that transforms our heart through our own effort and endeavour. Taking the Three Refuges actually does that to a degree. Whenever we take the Three Refuges and really bow inwardly to the Three Refuges -- Buddha, Dhamma, Sa\.ngha -- that symbolises the awakening heart in yourself; the Buddha, the truth of the teaching of the Buddha, and those who have liberated themselves. Each time we take refuge in something that is vaster than our little mind. Instead of taking refuge in `me', `mine', what upsets `me', what is self-obsessive, I take refuge in my awakened heart, in the truth within myself that opens my heart and mind to the bigger picture.

Have you noticed how when we look at ourselves we keep bumping into our obstacles? That is why the practice can feel quite frustrating sometimes if we don't have somebody experienced who can explain to us that obstacles are actually okay. Feeling wretched, undermined, miserable, and all that sort of thing is fine because these are only states of mind, perceptions that are impermanent. 

Naturally, the backdrop of those things is not always clear. That is why mindfulness is cultivated, because it's the backdrop. Sometimes, those things are very deeply rooted in our mind. It's not easy to uproot them and let go of them. Sometimes it may take years of witnessing particular patterns or particular responses to life before being free of them. Everything in you knows not to hold on to them, yet we have other emotional aspects that are preventing the process of letting go. But we have enough knowledge and understanding to realize that although our mind may be feeling stuck, a great chunk of ourselves is not stuck at all and feels fine. 

To be able to turn around and take refuge in the bit that is not stuck is an art and a skill. On this spiritual path we are able to keep looking at the part of ourselves that is already free, and take refuge in that. Of course, we need all the help we can get. It is very fortunate to have a good teacher such as Ajahn Sumedho to whom I can go to for advice, and who is not necessarily going to pamper me or tell me how good I am but who can remind me to stay mindful, wakeful, and very present with things that may be quite difficult, or even unbearable. 

Whatever arises in the mind, just don't cling to it. If you do this often enough with whatever arises, with difficulties and problems, it really works. Something shifts and is transformed. Your world changes and as your mind gains more and more confidence in the realm of Dhamma, the Truth. It loses its trust in your desires and fears.

A lot of our inability to let go comes from fear. We are frightened of letting go of things because everything we know, even our miseries, is comforting on an emotional level -- it's better than not knowing. Ajahn Sumedho taught us for many years to train the mind to face the unknown, and when questions arise just say, `I don't know \ldots{} I don't know \ldots{} I don't know.' Training the mind just like that.

Do that in your everyday life. Allow the Dhamma to inform your consciousness, rather than continuing on the treadmill of the conditioned mind's activities. All the conditioned mind can do is to go from one thought to another. It's not that there is something wrong with the thinking mind. The thinking mind is useful for contemplation, for reflection, for clarification and living your everyday life. As you contemplate the space of your mind, you can look at thoughts not as belief systems, but just as energy, as images, forms. Then you can make clear what it is you want to consciously think, and what you don't want to think. 

The path of practice is divided into three aspects; \textit{s\={\i}la} or ethics, \textit{sam\=adhi} or the practice of meditation that includes effort or energy, concentration, and mindfulness, and \textit{pa\~n\~n\=a}, or wisdom -- Right Understanding and Right Intention.

\looseness=1
This practice leads to a lot of joy, happiness, and peace. But as long as it is dependent on something, it is going to change. So we can't count on something that depends on impermanent causes. This practice is leading you to understand a mind that is in many ways very treacherous and tricky. To relate to ourselves and to our mind in a sound, sane, kind, and patient way in the face of this trickiness and delusion is a real skill. It's a training, it's an education, it's something we do little by little. It doesn't come by itself. We must learn to really take care of our actions -- body and speech and mind.

Most of us start with the mind. We get interested in meditation and then notice how angry we can get, or we notice the \textit{kilesas}, the afflictive emotions, or unskilful mental states that are very unpleasant. Then, we notice our attachment. Even being attached to being a good person is painful because it is going to blind you. There are a \textit{lot} of elements in ourselves that are blinding. We unravel these things when we are ready to open ourselves to our life fully, fearlessly. You can only do this if your goal is very clear. If you reflect on why you started on this path in the first place and what you want to do with it, or whether you really want to be free, you would be surprised to hear that perhaps you don't want to be free and that you just want your piece of cake and to eat it! Hearing that voice is enough. You don't have to believe it because it's not you. 

At some point, you have to be very clear that this path of practice is for the sole purpose of freeing the heart from misery, \textit{dukkha}. So, when you experience \textit{dukkha} don't shy away from it because this is your opportunity. It's not a problem, it's your opportunity to liberate your mind from its attachment to ignorance. It's what you are supposed to see. 

I leave you with this.
