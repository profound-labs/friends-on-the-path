
% Taking Refuge
% From a talk by Ajahn Sundara

After three or four days on a meditation retreat we tend to look a lot brighter and happier than when the retreat began. This is the natural outcome of looking inwardly and being present with what is happening in ourselves. However horrible things might feel, as we listen closely to our heart and mind some lovely things happen, we begin to relax. It's not easy, yet we begin to be more accepting of all the pain, of the suffering that we usually tend to put aside. We never seem to have the time and the space to live in harmony with ourselves. So when we go on retreat what a wonderful opportunity to be able to listen, and perhaps to understand a bit more profoundly the nature of our mind, the nature of our thoughts, of our feelings and perceptions. We have the chance now to realize that we only feel limited and bound by them \mbox{because} we rarely have the opportunity to pay attention to, or investigate and question their reality.

Before giving a talk, we traditionally take refuges in the Buddha, the Dhamma and the Sa\.ngha. When we become a monastic, a homeless one, we trade our home and we get Three Refuges. So we're not totally homeless. We actually take three very secure refuges and we leave behind all that we suppose to be safe, that we assume to be protective and secure. We leave behind home, family, money, the control of our lives, the control over the people we live with, the place we actually stay -- we let go of all that. And in return, we take the Three Refuges. 

Now, in my experience these refuges did not mean very much at first. I didn't quite understand what they were about. Several times a year, we have Buddhist festivals and ceremonies. We follow a lovely custom on those days. We meditate through the night and before the all-night vigil we slowly walk around the monastery three times holding a candle, some incense and some flowers in our hands. Monastics and lay people walk together silently around the monastery contemplating the Three Refuges; the Buddha, the Dhamma, and the Sa\.ngha. It's a very beautiful and inspiring sight. At first I didn't know what this really meant. I knew somehow that those refuges were in the human heart and perhaps as I practised I would come to know what they meant. I felt at that moment that I had a whole lifetime to understand this, so I just relaxed considering that the Buddha, Dhamma, Sa\.ngha wasn't something I had to think about.  

I think what brings many of us to be interested in the practice of meditation is the need to understand ourselves, the need to clarify the confusion we live in. Many of us want to be free; we want to understand and see for ourselves what this life is all about. At some point we get fed up with books; we've read enough, we've listened enough, we've met enough wise people. We've done everything we could to understand, and yet that didn't seem to be sufficient. Second-hand knowledge somehow is not really satisfactory. We want to experience for ourselves what all these wise people and all the wise teachings are saying. As long as there's no realization of the nature of our mind there's no real understanding. It's difficult to taste the joy and the freedom of knowing and experiencing the Buddha's teaching for oneself -- what's known as insight, seeing directly into the nature of our mind and body and realizing the freedom experienced when we let go of any attachments. At the beginning of the path of practice we still tend to look for a form of happiness. We all want to be happy, don't we? Who wants to be miserable? We all want to feel free and to experience pleasure.

The practice is not here to make us suffer. We only suffer because we haven't practised properly yet, we haven't done what is necessary to let go of ignorance, to let go of our attachments. So it's important to take this into account. We should not imagine that because we are practising we have to be terribly serious and feel that unless we experience some terrible pain or hardship that somehow something is not quite right. But more often than not it's true that, unless it hurts, our ignorance is not acknowledged. If it doesn't hurt, we can go on forever without really being aware of it. This seems to be our human predicament. Unless something hurts, we don't really wake up, we don't open our eyes and look. So everyday we recite the Three Refuges as a reminder, because out of habit we tend to take refuge in things like anger and worry, self-pity, pleasure and distraction. So our tendency is to take refuge in the wrong things, things that make us unhappy. And if we didn't have reminders, if we didn't have skilful means to bring back into consciousness what's really important in life, we would forget ourselves and never see the way out of suffering.

\section*{Refuge in the Buddha}

The refuge in the Buddha is the refuge in the knowing. The Buddha knows the world -- which in Buddhism does not mean the world of mountains, rivers and trees, but the world that arises in our mind and body, and the suffering that we create out of ignorance. In our daily chanting we say that the Buddha knows the world, he knows the arising of the world, the ending of the world, he knows the way the mind creates the realities we live in and through taking refuge in mindfulness, in the 'one who knows', we begin to see clearly the path that leads us out of suffering.

Somebody was asking me today, `Who is the one who knows? Who is the one who is aware?' A good question, isn't it? Because I have never found anybody `who knows', have you? I tried for a long time and finally gave up, and ever since I have made peace with the fact that there is nobody who knows. Just knowing. Knowing seems to be able to carry on functioning with or without my doubts. Without having an answer, I can still take refuge in being the `knower', being the one who's aware, who can see. The `One who knows' is that factor that balances out the extremes of  the mind. We can see the extremes of the mind, happiness, unhappiness, pleasure and pain, inspiration and despair. We can see hope and depression. We can see praise and blame. We can see agitation, sleepiness, boredom, the whole lot. And that seeing is a balancing factor, because we become aware of our attachments to these moods, these states of mind. Without a refuge in the knowing, in the awakened mind, we'd never be able to look at the mind. We'd be lost in confusion. So the refuge in knowing is very important. Together, the refuges are called the Three Jewels -- and they are really like beautiful jewels that we can go back to whenever there is confusion, whenever there is agitation. We can always go back and take refuge in knowing those states. We don't have to think about them, we don't have to psychoanalyse ourselves. We can actually go back to the knowing. And what happens then is that we see what the Buddha saw: impermanence. We can see that these states are not worth holding onto because they are impermanent and not satisfactory. And we get the intriguing feeling that maybe we are not `This.' Maybe it's got nothing to do with `Me.' Maybe my depression is not `My' depression.

\looseness=1
Wouldn't it be wonderful to realize that one's sadness is actually not a personal thing? Because we tend to think that everything that happens to us is personal, we create many problems in our lives: `Poor me, I'm the only one that this happens to. No one else has this problem, except me.' Everyone else looks terribly confident, don't they -- especially if we lack confidence in ourselves. Everyone else seems to be terribly strong and seems to really know what he or she is doing. I used to think like that. I used to look at someone and, if I felt a bit depressed or miserable, I could be quite convinced that they were okay. They were fine. I was the only one who had problems until I realized they, too, had problems. Because we are self-centred creatures by nature, everything is `my' problem, `my' life, `my' sorrows, `my' relationships, and 'my' melodramas. Everything seems to centre around `Me.'

Refuge in the Buddha allows us to see this very clearly. And it's a compassionate refuge. It's not a refuge that's critical. When we take refuge in mindfulness, we don't have to criticize or condemn or get angry with ourselves. We can observe the tendency to be critical, angry or demanding towards ourselves. This refuge in the Buddha is described in one of the first lines of our chanting as; `the Buddha absolutely pure, with ocean-like compassion' and that's really what that refuge means. It is a beautiful, compassionate home.

So we have three homes, three refuges. We have refuge in the Buddha. It doesn't have a roof, no central heating, but it feels very good. It feels very secure, very reliable -- especially when you see how much of our life is so agitated, so unreliable and insecure. As we become more aware, we have a clear view and a clear understanding of what \textit{sa\d{m}s\=ara} -- the endless round of birth and death -- is all about. And we are all here to get free from our attachment to it.

Much of our struggle in life is about trying to create situations where 'my' personality won't have to face suffering, or endure pain, won't feel embarrassed or ashamed. However every time we get lost or are unkind, angry, impatient or stupid, we can remember to be aware without judgement. We can acknowledge what is happening and accept it as it is. As soon as we have this clear vision of what's going on, we realize that our experiences are changing, and see clearly the uselessness of struggling to keep them permanent. We have to learn to be aware, to have mindfulness (\textit{sati}) in our heart as a refuge that protects us, that protects the heart.

\section*{Refuge in the Dhamma}

The second refuge, the Dhamma, is very close to the first one. In fact, there is a famous teaching that the Buddha gave to his disciples just before dying. They were anxious about him leaving this world and wondered who was going to be their teacher after the Buddha's passing away. They were concerned as to who was going to take over and be their guide. And he said: `The Dhamma and the Vinaya will be your guide and your refuge.' On a previous occasion he had also said that: `Who sees the Buddha sees the Dhamma, who sees the Dhamma sees the Buddha.'

Dhamma and Buddha -- there's no need to have a physical Buddha. We can actually find the Buddha, the one who knows, the one who is aware, in our own heart. And as soon as we are aware, mindful, we are in touch with the Dhamma. That's the beauty of this practice.

Sometimes, when we read books about Buddhism, we think we have to read the whole \textit{Tipi\d{t}aka}\footnote{\textit{Tipi\d{t}aka}: The Buddhist P\=a\d{l}i Canon.} before we can get in touch with the Dhamma. We believe that we have to learn the \textit{Abhidhamma},\footnote{\textit{Abhidhamma}: A philosophical framework based on the Buddha's teachings.} perfect the ten p\=aram\={\i}tas, develop the five spiritual faculties, get rid of the five spiritual hindrances and know the 56 states of consciousness, and so forth. By the end, we can feel so exhausted that we don't even want to start. In fact, today I was reflecting that when in our meditation period we mindfully breathe through our nostrils enduring a little bit of pain, a little bit of sweating or bearing with the heat and the cold, noisy people or boredom, we haven't got any idea of the amount of things we're really practising with. We don't know yet that at those moments we're perfecting the ten p\=aram\={\i}tas, that we're letting go of the hindrances and developing the five spiritual faculties of faith, effort, mindfulness, concentration and wisdom. We might not be aware of it but we're really perfecting many spiritual qualities of the heart. But it doesn't seem like very much, does it? We're just breathing in through the nostrils and then breathing out, and then we feel a bit of pain, then it's gone. Nothing much really. And yet over some years of practice we begin to see the fruits of our effort, and the teachings come alive.

So the refuge in the Dhamma is not something we have to look for very far. We don't have to look for the Dhamma somewhere out there, in another country, or in another person, or for a thing that will happen tomorrow or next year.

The quality of Dhamma is immediacy (\textit{sandi\d{t}\d{t}hiko}) -- right here, right now.  It is timeless (\textit{ak\=aliko}). The Dhamma invites us to `come and see' (\textit{ehipassiko}), it leads inward (\textit{opanayiko}) and can be realized when by oneself (\textit{paccatta\d{m} veditabbo}) when awareness and wisdom are present. Each morning we chant those qualities. We don't have to wait for someone to tell us what it is. We don't have to read books or go through a progressive step-by-step study before we can get in touch with Dhamma.

The refuge in awareness brings us into the present and in the present we can see the Dhamma, the truth of the way things are. But this can only be seen when there is a clear awareness of the present moment, and a seeing of the nature of our mind and the way it functions. We can notice that it has its seasons and cycles, its time of darkness and brightness, its own rhythm. And because we are often unaware of this rhythm, we can sometimes drag ourselves to the point of complete exhaustion, sickness or stress and forget that we are part of nature, part of `the way things are'.

Our intelligence, our capacity for knowledge, tends to alienate us from our nature. We often feel estranged from \mbox{ourselves} because our human nature is not really that exciting. Thoughts are so much more exciting! We can think, think, think the most wonderful things and the most miserable ones and our imagination can be quite creative, especially on retreat. We can really see how the mind is this wonderful creator. A famous Thai meditation teacher said once that in Buddhism it's not a God that creates, it's ignorance. We create out of ignorance. Sometimes we wonder what we have done in the past because our mind can think of the most bizarre things. We tend to have a lot of ideas of how things should be, how we would like things to be, how we think things should be, but have very little space for `the way things are,' for what is happening in the moment as it is. In fact, after a while one can see a really clear pattern in the mind: there is what we think it should be, then there is what we'd like it to be, and finally what is. All three seem to have a bit of a hard time cooperating with each other. In my early years, it took me a while to notice this pattern but through the practice, I began to understand that in one moment we can only be aware of so much -- which is often not very much. We can think a lot of things but we can actually know only a little. It's through knowing and investigating that which we are, that understanding deepens.

As long as we take things personally, we miss the Dhamma and are fooled by what arises in our mind. We fail to see that the things that we are taking personally are not what we are, nor what we think they are. We tend to believe and identify with the constant stream of thoughts, feelings and perceptions of our mind and it's no wonder that we become neurotic.

It is a matter of practising with right attitude, with an attitude of compassion and infinite patience, rather than developing and perfecting any particular techniques. Because although we may have done a lot of practice and be an expert in breath meditation, body sweeping and all that, if we are still striving to develop the perfect \textit{\=an\=ap\=anasati} meditator our approach is wrong. Without a correct perspective, we are still caught up with the idea that we have to improve on `Me.'

The immediate and direct nature of the experience of Dhamma is something quite extraordinary. We can realize the nature of our thoughts without any intermediary, without interpretation, and see them just as they are. It is quite a remarkable thing and it's what attracted me most to this teaching. When I came to practice I was overjoyed at the simplicity and immediacy of the realization of the nature of the mind. You did not have to learn too much or get a Ph.D., you didn't have to start accumulating more knowledge. In the practice of Dhamma, there is a process of letting go, of emptying and freeing ourselves from the burden of knowledge, from the burden of accumulated experiences, from the heaviness of being somebody or carrying a person in the mind. I remember that when practising in the world as a layperson -- now of course this is not to influence you all to become monks and nuns -- I had the feeling that I was always `somebody' practising. I found that very difficult. There was this burden of `me' practising. When I came to the monastery I was \mbox{ordinary} and could forget about feeling 'special', some strange creature on the spiritual path, because everybody in the monastery was doing the same thing, you were just normal. 

Much of our training in the monastery focuses on the ordinary. Daily, we spend periods of time cleaning, sweeping, dusting, walking from one room to the next, doing simple jobs and paying attention to the most mundane things such as opening doors, getting dressed, eating, getting up in the morning, brushing our teeth, putting our shoes on, going to the toilet, going to bed. Simple things like these are not exciting, and our mind learns to calm down and be more simple, more ordinary. We can see our mind wanting to make things special. If I had not been living in the monastery, I would never have seen the way the mind can create melodramas out of absolutely nothing. To be in touch with the ordinariness of our life is something very difficult for us because we have been conditioned to get our boost of energy through things that are interesting or stimulating. Or, we focus our attention on the next thing -- on what's going to happen next.

Unless we have guidance and help from a wise teacher, from wise people who have an understanding of the path, we tend to carry on in our spiritual practice in the same way as before we started. We're still looking for fascination, for excitement, for something special, for the big bang, for the flashing lights, for the super insight that's going to solve all `my' problems. I'm afraid it doesn't work like that. With the practice there is a change in our relationship with our mind. We let the flux of greed, hatred and delusion flow. We don't make a problem about it any more. We let the flow of our own mind just take its own course. We stop shaping the flux of our thoughts and feelings into this or that. Being in harmony with Dhamma is making peace with whatever is going on now, with `the way things are', the Dhamma.

It's difficult to be ordinary and accept the triviality of our life. That's why most of the time we feel frustrated, because we think that somehow things are going to be different, or that they should be different, don't we? We sense that life shouldn't be just getting up in the morning, having breakfast, getting bored, having a cry with one's spouse, going to the toilet, eating, getting bored at work, coming back, watching television, going to bed, getting up in the morning, and on and on and on, day after day after day. We feel that somehow there must be something else. So we go on a trip and travel around the world -- and we find out that even on the other side of the world, we still have to get up, we still have to go to the toilet, we still have to eat, we still get happy and bored with ourselves, we still get annoyed and depressed. We still get the same old `me' -- whether we are here, or in California, or in India, or anywhere. To come to terms with that has been the greatest teaching of monastic life.

Actually, monastic life is externally quite repetitive and boring. And if we identify with the structure or the routine then it's the most tedious lifestyle. It's so monotonous at times, you have no idea! But through accepting the perception and feeling of boredom for example, we realize that it's actually quite okay. It is not so much a matter of getting rid of boredom but of seeing what we are expecting from life. I spent many years expecting from life something it could not give me.  And in the same way if I expect something from the monastic life that it cannot give me, then I'll be very disappointed, frustrated or in a constant state of conflict.

So seeing the way things are is a very important realization because then we can actually work with life as it is, rather than expecting or dreaming about it. Expectations are like dreams. And most of our life is like a dream, or like a cloud, and we hope that this cloud will give us something real and substantial. Have you ever been able to shape a cloud? Or a dream? Yet this is what we are always trying to do isn't it? 

So there's this dreamlike state that we create out of expectations, out of not understanding the limitations of our mind and body, of our life and the world we live in. But if we see those limitations for what they are, then a wonderful thing happens; we can actually work with life as it is. We don't have to expect something from it any more. We can actually give to our life. And that's a great change in the mind. Through the practice we begin to see that we don't have to ask or get or demand something from life. We can actually give, offer and joyfully respond to it. And this, we can all do.

The natural process of the realization of Dhamma is the awareness that life is a constant opportunity to give, to be generous, to be kind, to be of service in whatever situation we are in. As we let go we don't get so caught up and obsessed with ourselves. We can actually be useful. We can help. We can give. We can encourage ourselves and the people around us.

\section*{Refuge in the Sa\.ngha}

The refuge in the Sa\.ngha, the last one, is the refuge in noble friendship -- \textit{kaly\=a\d{n}amitta}. It symbolizes the community of men and women, ordained or living in the world, who have taken refuge in living wisely and compassionately, in accord with the Dhamma. They take refuge in harmlessness, loving-kindness and respect for all living beings. These are people who have a moral conscience. They are aware when they are acting foolishly or harmfully. This refuge symbolizes the purity of the human heart. I remember when for the first time I heard of the concept of the `Pure Heart.' I thought that it was a beautiful expression and that's really what that refuge is; it's a refuge in that in us which is good, wholesome, compassionate and wise.

Before I started being interested in Buddhism, I used to do short retreats in Christian monasteries.  The thing that struck me most in those places was this awesome, pervading feeling of respect for life and for each other. Even the silence seemed to be a kind of acknowledgement of reverence, of honouring the best in human beings. It was very moving. Even though I could not explain what it was, I sensed that people were devoted to something good and true. When I came to Chithurst and met the community for the first time, I had a very similar feeling of meeting human beings totally dedicated to honouring the truth, to being it and living in accordance with it.

And so the refuge in Sa\.ngha was the first thing that brought me to the monastic life.

My interest in joining the monastic Sa\.ngha came from the need to have a vehicle and a refuge of sanity in myself that would provide some guidance. I realized, for example, that without an ethical standard to contain and understand the energy of my desires, I was really in trouble. I was always very good at knowing what I should do, what I should be; I was a real expert at creating ideals! But somehow the energy of my desires had a very different agenda. My self-gratifying habits on the one hand and my yearning for truth on the other didn't meet, didn't seem to be very good friends.

One of the first things that became really clear when I joined the Sa\.ngha was that the precepts were my best friends and my best protectors. I never had the feeling that they were imposing themselves on me at all. On the contrary, I knew that they were supporting me and reminding me of being more mindful of my speech, my actions and my thoughts.

The training of our body and mind requires an enormous amount of patience and compassion. Our habits are strong and if we have lived a fairly heedless life in the past, we can't expect to turn instantly into a virtuous person. When we arrive at the monastery we don't become a saint overnight! And it is not a meditation retreat and the keeping of the precepts for ten days that is going to turn us into one either, is it? But at least we have a situation and a teaching that can help us to look at what is not correct or skilful in our behaviour and our habits. We learn to make peace and see them as they are. So we take refuge in the Sa\.ngha and use the standards followed by those who have walked the Path and liberated themselves before us. This refuge points to our commitment to virtuous conduct, to a way of life that protects and nurtures peace in the heart and reminds us of our intention to liberate it from ignorance. If we didn't have these guidelines, we would easily forget ourselves. And we are very good at that. In fact, that's what the mind is most intent on and does all the time, it forgets. But when we take refuge in mindfulness, in the Dhamma and in the purity of our intention to free ourselves from delusion, we remember that we have the necessary tools to train the heart. We can see clearly the unskilfulness of our habits, of our speech or of our thoughts and so on.

These refuges may appear as if they were three: Buddha, Dhamma, Sa\.ngha. But actually they are just one. We don't have one without the other. When there is virtue and the intention to live harmoniously, with compassion and respect for oneself and each other, then there's a naturally growing awareness, in harmony with the Dhamma, and we are more attuned to the truth. All of them interact and affect each other.

At first, we don't know quite what or where these refuges are. They may seem to be just words. You might even feel confused and have no trust in them. But as we practise, as we keep letting go of our attachments to thoughts, feelings, perceptions, they become a growing reality.

We can actually experience these refuges. They become a part of our life, a part of something that we can go back to, right here, right now. We don't have to wait. They are always present in our heart. Here, now, in the present. That's the real beauty of the practice of the Path. It's that total simplicity, that immediacy, complete in itself. There's nothing else that you need. 

Just in taking the Three Refuges, you've got all the tools you need for your heart to be free.
