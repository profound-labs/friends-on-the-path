
% The Knowing Mind
% Ajahn Sundara

\vspace*{0.4\baselineskip}
In this practice we develop that quality of mind which knows -- the `knowing' mind. Not `reacting' mind, but the `knowing' mind. It's very different. Within the quality of knowing you can see everything. You can see the reactions. You can see the pain. You can see the joy. You can see the peace. You witness everything. 

This knowing does not fight what's going on, or try to change what's going on, or make something out of what's going on.  As we make peace with everything  that is happening now, the quality of knowing can strengthen itself, enabling you to clearly mirror what is going on. The knowing mind can mirror everything that's happening. This is the path of knowledge, the path of knowing, the path of understanding. 

Know change. Know that comfort and discomfort, pleasure and pain, succeed each other naturally. You don't need to pick them up all the time. Just know that much. When there is a pleasant, comfortable, bright mind, you know that certain conditions have brought this about. The opposite is true too -- when there is discomfort and confusion in the mind, other conditions have brought this about. \textit{You} haven't done it. It just \textit{happens}. You think \textit{you} do it, but actually, this is just the Dhamma, this is just nature. And if you stop picking up this high and low, pleasure and pain cycle, the mind settles down and becomes more peaceful. Peace and happiness don't come from picking up one side of reality and avoiding the other. Peace and happiness come from not picking up anything at all.

Let go. As you sit, the mind that knows is powerful. This is the power of the knowing, the capacity of the knowing, the breath of the knowing mind, this is our refuge. This is what we call `The Buddha Refuge', the One Who Knows.

Just be the witness of what you are. It doesn't seem like much, but it's completely transformative. It's like comforting an old friend who has been in trouble -- not mistreating him or her. We bring loving-kindness to the experience of this mind, this body, this person here who may be struggling or confused about what is going on with their life. Make peace with this old friend. 

The knowing mind is not necessarily an experience that we are familiar with. The experience of the six senses is a familiar experience. When you say `hot and cold', `pleasant', `lovely', `delightful', `delicious', you know what it means. There's a kind of familiarity about our sensory world that is relatively easy to pin down. But the knowing mind is a strange thing. You can't put your finger on it. It has a mirror-like quality -- a `witness' quality that enables you to know yourself and to know the nature of your mind and body. 

The Buddha's teaching helps us investigate the nature of mind and body in the light of three characteristics -- \textit{anicca, dukkha, anatt\=a} (impermanence, pain or suffering, and not-self). These are very abstract concepts at first, but as you begin to bring the knowing mind into your experience it becomes utterly obvious. Pain comes and goes. Qualities of the mind such as happiness, dullness, expansiveness, tightness, and concentration, come and go. This is what you need to look at carefully in your meditation -- to see change. It's a learning process. Learn to see \textit{dukkha} as a direct experience, not as a conceptual perception.

Learn, also, to see \textit{anatt\=a}, not-self, as a direct experience. Notice that \textit{you} don't have much control over what you are experiencing. Whatever arises passes away and is not really under your ownership. You don't own this mind and body. If you owned your mind you could tell it to do this or do that, and it would do it. `Don't be miserable!' and you would not be miserable. `Wake up!' and you would wake up.

Although we don't have control over what we experience, there is still a certain amount of decision that we can make. We can decide to sit upright, and to be attentive -- we can \textit{intend}. Intention is very powerful, so there is some element of intention. You can \textit{intend}, but things might not happen the way you expect, or the way you imagine, or the way you anticipate. You just \textit{intend}. You decide to be awake this morning -- maybe you will, maybe you won't. That's okay. Just still be the witness. Sometimes you are awake. Sometimes you are asleep. At least your intention is very clear. You can make it clear. This is as much as you can do, just intend, and then witness the result of that intention.

Intention is very powerful, especially in the morning when the mind is still very malleable, soft, not yet caught up in all the wilfulness of worldly pressures. That's why people say it's very important to sit in the morning and prepare your heart and mind for this day. One doesn't often realize the importance of starting the day with a clear intention to see the way things are, to acknowledge what's happening in ourselves, and to do what we have to do, no matter whether we feel lazy, or we feel elated, inspired, or dispirited. If there is something we need to do, we just do it. 

Maybe you have a hard time getting up in the morning. Just make that conscious. Maybe you are a night owl and you prefer waking up late, you find early mornings difficult. You don't need to fit your mind's image of the perfect Buddhist meditator -- just witness the struggle of your mind, or the discomfort. If your mind is bright and comfortable in the morning then notice that too. The more you witness and take refuge in the knowing mind, the more you will find that whatever needs to be discarded will be discarded. Whatever is not useful will just naturally fall away, but not through rejection or aversion -- simply through the knowing of this experience in each moment. 

The knowing mind will enable you to remove obstacles. As you continue your meditation practice and get in touch directly with the three characteristics of existence, you will gain confidence in another whole reality of yourself which is not dependent on attachment and fear and blind desires. This is the fruit of the Dhamma practice; strengthening the refuge of knowing, strengthening the awakened mind, strengthening our confidence and trust in the mind that knows. The more you trust it, the more you will experience that life goes on much better without an identity, a self, an attachment to \textit{dukkha}, and an ignorance of \textit{anicca}. Life goes on much better with knowing impermanence, knowing pain for what it is and letting go of the habitual tendency to think you are in charge.  And as you practise, trust increases -- trusting in the Dhamma, trusting in the knowing, trusting in your intention.

In the world, we need to keep our intentions clear. We need to use thoughts and develop clarity of intention. The power of our mind doesn't lie in attaching to habits. The power of our mind comes from letting go of fear, from letting go of delusion, and increasing clarity. Let go of control, let go of habits, and take on board the power of clarity that makes you intend things in the right directions. Intend things so that you increase the happiness in your life. Intend to develop that which is good and refrain from doing evil. By intending with clarity, the development of the mind becomes a vital, essential practice.

