
% Why Come to a Monastery?
% Ajahn Candasiri

One question we all need to ask ourselves is, `Why do we come to a monastery?' Whether we are monks, nuns, novices, lay guests or visitors, we should ask, `Why have I come?' We need to be clear about this in order to derive the greatest benefit from what a monastery has to offer. If we are not clear, we can waste a lot of time doing things that may detract from the possible benefits to be found here.

The Buddha often spoke of three fires -- three ailments -- that we, as human beings, are afflicted by. These three things keep us continually moving, never able to rest or to be completely at ease. They are listed as greed, hatred and delusion (\textit{lobha, dosa, moha}). He also, out of compassion, pointed out the antidote.

Actually, these fires are based on natural instincts. For example, greed, or sensual desire is what enables humanity to survive, whether it be the desire for food or the sexual drive. Without sexual desire, none of us would be here now \ldots{} and, of course, without hunger, which is the desire for food, we would not be inclined to take in the nourishment we need to maintain the body in a reasonable state of health. However, a difficulty arises when we lose touch with what is needed or necessary, and seek sensual gratification for its own sake.

Another kind of survival instinct is our response to danger. Either we turn around and attack something that is perceived as a threat to our physical survival, or we try to get away from it. This is the basis for \textit{dosa} -- hatred or aversion. Clearly, this also has an important place in nature, but again we have become confused, and what we frequently find ourselves defending is not so much the physical body but the sense of self: what we perceive ourselves to be, in relation to another.

The third fire, which follows on quite naturally from this, is delusion -- \textit{moha.} We don't really see clearly or understand how things are; we don't really understand what it is to be a human being. Instead, we tend to fix ourselves and each other as personalities, or `selves'. However, in fact, these are just ideas or concepts that we measure against other concepts of who or what we \textit{should} be. Then, if anyone comes along and challenges that `self', it can invoke a strong reaction -- we instinctively attack, defend or try to get away from the perceived threat. Really, it's a kind of madness, when you think about it.

Now, as I said before, the Buddha, having pointed out the nature of the disease, also presented the cure. This came in the form of simple teachings which can help us to live in a way that enables us to understand, and thereby free ourselves from these diseases; they also help us to avoid doing things that exacerbate them.

This brings me to what I would suggest is the best reason for coming to a monastery. It could be said that the best reason for coming to a monastery is because we want to free our hearts from disease, from the bonds of desire and confusion; and we recognize that what is presented here is the possibility of bringing this about. Of course, there could be many other reasons too: some people don't really know why they have come -- they just feel attracted to the place. 

So what is it about the monastery that is different from what happens outside it?

It is a place that reminds us of our aspiration and our potential. There are lovely images of the Buddha and his disciples, which seem to radiate a feeling of calm, ease and alertness. Also, here we find a community of monks and nuns who have decided to live following the lifestyle that the Buddha recommended for healing those diseases.

Having recognized that we are sick and that we need help, we begin to see that in fact the cure is in direct opposition to the ways of the world. We see that if we are to cure ourselves, we need first to understand the cause of the sickness -- which is the attachment to desire. So we need to understand the three basic kinds of desire: to get, to get rid of and to exist and be a separate self -- in order to free ourselves from the attachment to them. Instead of blindly following desires, we need to examine them closely.

The discipline we follow is based on precepts -- simple guidelines outlining the practice of responsible living. These, used wisely, can engender a sense of dignity and self-respect. They restrain us from actions or speech that are harmful to ourselves or others. They also delineate a standard of simplicity or renunciation, whereby we ask: `What do I really need?' instead of simply responding to the pressures of a materialistic society. But we may well wonder: `How do precepts help us to \textit{understand} desire?'

In a sense, what our monastic discipline offers is a strong container within which we can observe desire as it actually arises. We deliberately put ourselves into a form which prevents us from following all of our desires. This allows us to see desire itself, and to notice how it changes. 

Normally, when we are caught up in the process of desire there is no sense of objectivity. We tend to be totally identified with the process itself so it is very difficult to observe or to do anything about it, other than be swept along by it.

For example, with lust or aversion we can see that these are natural energies or drives -- which, in fact, everyone has. We are not saying that it's wrong to have sexual desire -- or even to follow it, in appropriate circumstances -- and we recognize that it is for a particular purpose, and that it will bring about a certain result. 

As monks and nuns we have decided that we do not want to have children. We also recognize that the pleasure of gratification is very fleeting, in relation to possible longer-term implications and responsibility. So we choose not to follow sexual desire. However, this does not mean that we don't experience it. It doesn't mean that as soon as we shave our heads and put on a robe, we immediately stop experiencing any kind of desire! In fact, what can happen is that our experience of these desires is enhanced when we come to a monastery. This is because in lay life we can do all kinds of things to make ourselves feel okay -- usually without really being aware of what it is that we are doing. Sometimes there is just a subliminal sense of unease, followed by the reaching out to get something to relieve it -- always moving from one thing to the next.

In the monastery it's not so easy to do this any more. We deliberately tie ourselves down -- in order to look at the drives, energies or desires that would normally keep us moving. Now you might ask: but what kind of freedom is this, tying ourselves down in a situation where we are constantly restrained, always having to conform? Always having to behave in a particular way; to bow in a particular way, and at particular times; to chant at a particular speed and pitch; to sit in a particular place, beside particular people -- I've been sitting next to or behind Sister Sundara for years! What kind of freedom is this?! It brings freedom from the bondage of desire. Rather than helplessly, blindly, being pulled along by our desire we are free to choose to act in ways that are appropriate, in harmony with those around us.

It's important to realize that `freedom from desire' does not mean `not having desire'. We could feel very guilty and really struggle if we thought like that. As I said before, desire is part of nature. The problem arises because it has been distorted through our conditioning: our education, upbringing, and the values of society. We are not going to get rid of it just like that -- just because we want to, or feel that we shouldn't have desire. It's actually a more subtle approach that's required.

The monastic form and precepts can help us to make a peaceful space around the energies of desire so that, having arisen, they can burn themselves out. It is a process that takes great humility because first we have to acknowledge that the desire is there, and that can be very humbling. Often, particularly in monastic life, our desires may seem extremely petty; the sense of self may be bound up in things that seem very trivial. For example, we might have a very strong idea about how carrots should be chopped; then, if someone suggests we do it differently we can become very agitated and defensive! So we need to be very patient, very humble. 

Fortunately, there are some simple reference points, or refuges, which can provide us with security and a sense of perspective amid the chaotic world of our desires. These of course are Buddha, Dhamma, Sa\.ngha: the Buddha, our teacher -- also that within us which knows things as they are, seeing clearly, not confused or agitated by sense impression; the Dhamma, the teaching or the Truth, how things actually are right now -- often quite different from our ideas about things; and Sa\.ngha -- the lineage or community of those who practise, and also our aspiration to live in accordance with what we know to be true, rather than following all kinds of confused and selfish impulses that can arise. The Buddha gave some simple ways of turning to these. These are called the Four Foundations of Mindfulness.

Mindfulness of the body is one I use a great deal in my own practice. The body can be a very good friend to us -- because it doesn't think! The mind, with its thoughts and concepts can always confuse us, but the body is very simple -- we can notice how it is at any moment. 

If someone acts or speaks in an intimidating way, I can notice my instinctual reaction, which is to tense up in a defensive attitude and perhaps respond aggressively. However, when I am mindful of the process, I can choose not to react in this way. Instead of breathing in, puffing myself up, I can concentrate on breathing out -- relaxing, so that I become a less threatening presence to the other person. If, through mindfulness, I can let go of my defensive attitude they too can relax, rather than perpetuating the process of reactivity. In this way, we can bring a little peace into the world.

People visiting monasteries often comment on the peaceful atmosphere they find here. But this is not because everyone is feeling peaceful, or experiencing bliss and happiness continuously -- they can be experiencing all kinds of things. One Sister said that she had never experienced such murderous rage or such powerful feelings of lust until she entered the Sa\.ngha! What is different in a monastery is the practice. So whatever the monks and nuns might be going through, they are at least making the effort to be present with it, bearing it patiently, rather than feeling that it shouldn't be like that, or trying to make it change.

The monastic form provides a situation in which often renunciation and constraint are the very conditions for the arising of passionate feelings. However, there is also the reassuring presence of other samanas. When we're really going through it, we can speak to an older, more experienced Brother or Sister in the life, whose response is likely to be something like, `Oh yes, don't worry about that; it will pass. That happened to me. It's normal, it's part of the process of purification. Be patient.' So we find the confidence to continue, even when everything seems to be collapsing or going crazy inside.

Coming to a monastery we find people who are willing to look at and understand the root cause of human ignorance, selfishness and all the abominable things that happen in the world; people who are willing to look into their own hearts, and to witness the greed and violence that others are so ready to criticize `out there'. Through experiencing and knowing these things we learn how to make peace with them, right here in our own hearts, in order that they may come to cessation. Then, maybe, rather than simply reacting to the ignorance of humanity and adding to the confusion and violence that we see around us, we are able to act and speak with wisdom and compassion in ways that can help to bring a sense of ease and harmony among people.

So it's definitely not an escape, but an opportunity to turn around and face up to all the things we have tended to avoid in our lives. Through calmly and courageously acknowledging things as they are, we begin to free ourselves from the doubts, anxiety, fear, greed, hatred and all the rest which constantly bind us into conditioned reactions.

Here, we have the support of good friends, and a discipline and teachings to help keep us on course in what sometimes seems like an impossible endeavour!

May we all realize true freedom.

Eva\d{m}.
