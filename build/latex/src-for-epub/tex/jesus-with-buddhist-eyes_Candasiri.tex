
% Jesus with Buddhist Eyes
% Ajahn Candasiri

\vspace*{0.4\baselineskip}
His Holiness, the Dalai Lama, speaking to a capacity audience in the Albert Hall in 1984 united his listeners instantly with one simple statement: `All beings want to be happy; they want to avoid pain and suffering.' I was impressed at how he was able to touch what we share as human beings. He affirmed our common humanity, without in any way dismissing the obvious differences.

When invited to look at `Jesus through Buddhist eyes', I had imagined that I would use a `compare and contrast' approach, rather like a school essay. I was brought up as a Christian and turned to Buddhism in my early thirties, so of course I have ideas about both traditions: the one I grew up in and turned aside from, and the one I adopted and continue to practise within. But after re-reading some of the gospel stories, I would like to meet Jesus again with fresh eyes, and to examine the extent to which he and the Buddha are in fact offering the same guidance, even though the traditions of Christianity and Buddhism can on the surface appear to be rather different.

To start with, let me say a little about how I came to be a Buddhist nun.

Having tried with sincerity to approach my Christian journey in a way that was meaningful within the context of everyday life, I had reached a point of deep weariness and despair. I was weary with the apparent complexity of it all. Despair had arisen because I was not able to find any way of working with the less helpful states that would creep, unbidden, into the mind: the worry, jealousy, grumpiness, and so on. Even positive states that could turn around, and transform themselves into pride or conceit -- which were of course equally unwanted.

Eventually, I met Ajahn Sumedho, an American-born Buddhist monk, who had just arrived in England after training for ten years in Thailand. His teacher was Ajahn Chah, a Thai monk of the Forest Tradition who, in spite of little formal education, won the hearts of many thousands of people, including a significant number of Westerners. I attended a ten day retreat at Oakenholt Buddhist Centre, near Oxford, and sat in agony on a mat on the floor of the draughty meditation hall, along with about forty other retreatants of different shapes and sizes. In front of us was Ajahn Sumedho, who presented the teachings and guided us in meditation, together with three other monks.

This was a turning point for me. Although the whole experience was extremely tough -- both physically and emotionally -- I felt hugely encouraged. The teachings were presented in a wonderfully accessible style, and just seemed like ordinary common sense. It didn't occur to me that it was `Buddhism'. Also, they were immensely practical and as if to prove it, we had, directly in front of us, the professionals -- people who had made a commitment to living them out, twenty-four hours a day. I was totally fascinated by those monks: by their robes and shaven heads, and by what I heard of their renunciant lifestyle, with its 227 rules of training. I also saw that they were relaxed and happy -- perhaps that was the most remarkable, and indeed slightly puzzling, thing about them.

I felt deeply drawn by the teachings, and by the Truth they were pointing to: the acknowledgement that, yes, this life is inherently unsatisfactory, we experience suffering or dis-ease -- but there is a Way that can lead us to the ending of this suffering. Also, although the idea was quite shocking to me, I saw within myself the awakening of interest in being part of a monastic community. 

So now, after more than twenty years as a Buddhist nun, what do I find as I encounter Jesus in the gospel stories?

Well, I have to say that he comes across as being much more human than I remember. Although there is much said about him being the Son of God, somehow that doesn't seem nearly as significant to me as the fact that he is a person: a man of great presence, enormous energy and compassion, and significant psychic abilities. He also has a great gift for conveying spiritual truth in the form of images, using the most everyday things to illustrate points he wishes to make: bread, fields, corn, salt, children, trees. People don't always understand at once, but are left with an image to ponder. Also he has a mission -- to re-open the Way to eternal life; and he's quite uncompromising in his commitment to, as he puts it, `carrying out his Father's will.'

His ministry is short but eventful. When I am reading through Mark's account, I feel tired as I imagine the relentless demands on his time and energy. It's a relief to find the occasional reference to him having time alone or with his immediate disciples, and to read how, like us, he at times needs to rest. A story I like very much is of how, after a strenuous day of giving teachings to a vast crowd, he is sound asleep in the boat that is taking them across the sea. His calm, in response to the violent storm that arises as he is sleeping, I find most helpful when things are turbulent in my own life.

I feel very caught up in the drama of it all; there is one thing after another. People listen to him, love what he has to say (or in some cases are disturbed or angered by it) and are healed. They can't have enough of what he has to share with them. I'm touched by his response to the 4,000 people who, having spent three days with him in the desert listening to his teaching, are tired and hungry. Realising this, he uses his gifts to manifest bread and fish for them all to eat.

Jesus dies as a young man. His ministry begins when he is thirty (I would be interested to know more of the spiritual training he undoubtedly received before then), and ends abruptly when he is only thirty-three. Fortunately, before the crucifixion he is able to instruct his immediate disciples in a simple ritual whereby they can re-affirm their link with him and each other (I refer, of course, to the Last Supper) -- thereby providing a central focus of devotion and renewal for his followers, right up to the present time.

I have the impression that he is not particularly interested in converting people to his way of thinking. Rather it's a case of teaching those who are ready. Interestingly, often the people who seek him out come from quite depraved or lowly backgrounds.  It is quite clear to Jesus that purity is a quality of the heart, not something that comes from unquestioning adherence to a set of rules. His response to the Pharisees when they criticize his disciples for failing to observe the rules of purity around eating expresses this perfectly: `There is nothing from outside that can defile a man' -- and to his disciples he is quite explicit in what happens to food once it has been consumed \ldots{} `rather, it is from within the heart that defilements arise.' Unfortunately, he doesn't at this point, go on to explain what to do about these.

What we hear of his last hours: the trial, the taunting, the agony and humiliation of being stripped naked and nailed to a cross to die -- is an extraordinary account of patient endurance, willingness to bear the unbearable, without any sense of blame or ill will. It reminds me of a simile used by the Buddha to demonstrate the quality of \textit{mett\=a}, or kindliness, he expected of his disciples: `Even if robbers were to attack you and saw off your limbs one by one, should you give way to anger, you would not be following my advice.' A tall order, but one that clearly Jesus fulfils to perfection: `Father, forgive them for they know not what they do.'

So why did I need to look elsewhere for guidance? Was it simply that Jesus himself was in some way lacking as a spiritual template? Was it dissatisfaction with the Church and its institutional forms -- what Christianity has done to Jesus?  Or was it simply that another way presented itself that more adequately fulfiled my need at that time? 

Well, in Buddhism I found what was lacking in my Christian experience. It could be summed up in one word: confidence. I don't think I had fully realized how hopeless it had all seemed, until the means and the encouragement were there. There is a story of a Brahmin student called Dhotaka, who implored the Buddha: `Please, Master, free me from confusion!' The Buddha's perhaps somewhat surprising response was: `It is not in my practice to free anyone from confusion. When you yourself have understood the Dhamma, the Truth, then you will find freedom.' What an empowerment! 

In the Gospels we hear that Jesus speaks with authority; he speaks too of the need to have the attitude of a little child. Now, although this could be interpreted as fostering a child-like dependence on the teacher, Buddhist teachings have enabled me to see this differently. The word, `Buddha', means `awake' -- awake to the Dhamma, or Truth, which the Buddha likened to an ancient overgrown path that he had simply rediscovered. His teaching points to that Path: it's here, now, right beneath our feet -- but sometimes our minds are so full of ideas about life that we are prevented from actually tasting life itself!

On one occasion a young mother, Kisagotami, goes to the Buddha, crazy with grief over the death of her baby son. The Buddha's response to her distress, as she asks him to heal the child, is to ask her to bring him a mustard seed -- from a house where no one has ever died. Eventually, after days of searching, Kisagotami's anguish is calmed; she understands that she is not alone in her suffering -- death and bereavement are inevitable facts of human existence. Jesus, too, sometimes teaches in this way. When a crowd had gathered, ready to stone to death a woman accused of adultery, he invites anyone who is without sin to hurl the first boulder. One by one they turn away; having looked into their own hearts, they are shamed by this simple statement.

In practice, I have found the process to be one of attuning, of attending carefully to what is happening within -- sensing when there is ease, harmony; knowing also when one's view is at odds with What Is. I find that the images that Jesus uses to describe the Kingdom of Heaven explain this well. He says it is like a seed that under favourable conditions germinates and grows into a tree. We ourselves create the conditions that either promote well-being and the growth of understanding, or cause harm to ourselves or others. We don't need a God to consign us to the nether regions of some hell realm if we are foolish or selfish -- it happens naturally. Similarly, when we fill our lives with goodness, we feel happy -- that's a heavenly state.

On that first Buddhist retreat it was pointed out that there is a Way between; neither following, nor struggling to repress harmful thoughts that arise. I learned that, through meditation, I can simply bear witness to them, and allow them to pass on according to their nature -- I don't need to identify with them in any way at all. The teaching of Jesus that even to have a lustful thought is the same as committing adultery had seemed too hard; while the idea of cutting off a hand or foot, or plucking out an eye should they offend makes logical sense -- but how on earth are we to do that in practice? I can see that it would require far more faith than I, at that time, had at my disposal! So I was overjoyed to learn of an alternative response to the states of greed, hatred or delusion that arise in consciousness, obscure our vision, and lead to all kinds of trouble.

As the Dalai Lama said, `Everyone wants to be happy; no one wants to suffer.' Jesus and the Buddha are extraordinary friends and teachers. They can show us the Way, but we can't rely on them to make us happy, or to take away our suffering. That is up to us. 
