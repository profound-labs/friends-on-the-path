
We have put together this small collection of writings in response to an invitation to reprint some of the nuns' published teachings. There being no collection of teachings that was considered to be suitable for re-printing in its entirety, we have gathered material from various sources.

The Buddha used the simile of the great ocean which has one taste -- the taste of salt -- to point out that all of the Teachings (Dhamma) and Way of Practice (Vinaya) that he presented in his lifetime are concerned with just one thing: freedom from suffering. In the same way, although the teachings that appear in this collection were given over a significant time span and in different situations, the single intention is to point to causes of suffering and to offer suggestions as to how these causes can be eliminated.

The title of the book, `Friends on the Path', was chosen because it seems to be a theme in many of the teachings included here. We also recognize the central part that good friends (\textit{kaly\=a\d{n}amitta}) have played for each of us on our spiritual journey, and that the \textit{Dhamma-Vinaya}, the teachings and the training of the mind and body presented by the Buddha can also be seen as good friends and supports on the Path of liberation from suffering. We have been sharing this journey for more than thirty years and our continuing friendship through the ups and downs, disagreements and challenges of monastic life could also be seen as a testament to the efficacy of the Buddha's teachings and way of practice.

We would like to acknowledge our debt of gratitude to our teacher, Ajahn Sumedho, for his total dedication and untiring energy in pointing out the way of freeing the heart.
\bigskip

{\raggedleft\par
May all beings be free,\\
Ajahn Sundara and Ajahn Candasiri \par}
