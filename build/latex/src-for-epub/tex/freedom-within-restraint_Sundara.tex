
% Freedom Within Restraint
% Ajahn Sundara

When the Buddha taught the First Noble Truth, he said that taking refuge in human existence is an unsatisfactory experience. If one attaches to this mortal frame, one will suffer.

Not getting what you want is painful -- that's quite easy to relate to. Getting what you don't want can also be painful. But as we walk a little further in the footsteps of the Buddha, even getting what we want is painful! This is the beginning of the path of awakening.

When we realize that getting what we want in the material world is unsatisfactory too, that's when we start to mature. We're not children any more, hoping to find happiness by getting what we want or running away from pain.

\looseness=1
We live in a society that worships the gratification of desires. But there are many of us who are not interested so much in just gratifying desires, because we know intuitively that this is not what human existence is about. I remember many years ago when I was trying to understand what the Truth was, and I had no concept for it. I knew in a way that it was something beyond the reach of my thinking and emotional mind, something that transcended this world of birth and death.

As time went on, the desire to live a life that was truthful and real became the most important thing. As I was trying to harmonise my thoughts, my feelings and my aspirations and come to a place of peace, I became aware of the huge gap between my mind and my aspirations, what I would call `myself', and this body with its five physical senses. At the time, I didn't even realize that the Buddhist teaching presented human beings with a sixth sense, the mind, the platform on which thoughts can arise.

Mind and body are a reservoir of energy and I found that my energy fluctuated, depending on how I used them. My way of relating to life and my understanding of it seemed also to be dependent on the clarity of my mind, and in turn that clarity was very much conditioned by the degree of energy I had. So I was quite keen to find out how to live without needlessly wasting that energy.

Many of us have not been raised up with a very disciplined life-style. In my family I was brought up within an atmosphere that fostered a certain amount of freedom of expression. Yet, following our whims and fancies, doing what we want, when we want, doesn't actually bring much wisdom to our life, nor much compassion or sensitivity. In fact, it makes us rather selfish.

Despite not having been inculcated with any great sense of discipline, as a child I appreciated the beauty of being alive, the harmony of life, and the importance of not wasting it. Yet the idea of living in a restrained and disciplined way was quite alien to my conditioning.

\looseness=1
When I came across meditation and the practice of insight, it seemed a much easier introduction to discipline than following moral precepts or commandments. We often tend to look with alarm upon anything that is going to bind us, any convention that is going to limit our freedom. So most of us come to discipline through meditation. As we look into our hearts at the way we relate to the world of our senses, we come to see how everything is interconnected. Body and mind are constantly influencing and playing on each other.

We know well the pleasure involved in gratifying our senses when, for example, we listen to inspiring music or when we are looking at beautiful scenery. But notice how, as soon as we attach to the experience, that pleasure is spoiled. This can be very painful, and often we feel confused by the sensory world. But with mindfulness we gain insight into the transient nature of our sensory experiences, and become acquainted with the danger of holding on to something that is fleeting and changing. We realize how ridiculous it is to hang on to that which is changing. And with that realization we naturally recoil from wasting our energy on following that which we have little control over and whose nature it is to pass away.

Sense restraint is the natural outcome of our meditation practice. Understanding the danger of blindly following our senses, the desires connected with them, and the objects connected with the desires is one aspect of discipline. Understanding naturally brings about the application of this discipline. It is not sense restraint for its own sake but because we know that sense desires do not lead to peace, and cannot take us beyond the limitations of identification with our mind and body.

When we first come to live in the monastery we have to adopt the discipline of the Eight Precepts. The first Five Precepts point to what is called Right Action and Right Speech: refraining from killing, from stealing, from sexual misconduct, from lying and from taking drugs and intoxicants. The next three focus on renunciation, such as refraining from eating after a certain time, dancing, singing, playing musical instruments, beautifying oneself and from sleeping on a high and luxurious bed. Some of these precepts may sound irrelevant in our day and age. What do we call a high or luxurious bed today for example? How many of us have a four poster bed? Or why is dancing, singing or playing an instrument not allowed as a spiritual practice?

When we ordain as a nun or a monk, we take onboard even more precepts and learn to live with an even greater restraint. The relinquishment of money, for example, makes us physically totally dependent on others. These standards may sound very strange in a society that worships independence and material self-reliance. But those guidelines begin to make more sense when integrated into our meditation practice. They become a source of reflection and put us in touch with the spirit behind them. We find that they help us to refine our personal conduct and to develop a deep awareness of our physical and mental activity and of the way we relate to life. So that when we look into our hearts, we can see clearly the results and consequences of our actions by body, speech and mind.

Following such discipline slows us down, too, and requires that we be very patient with ourselves and others. We generally tend to be impatient beings. We like to get things right straight away, forgetting that much of our growth and development comes from accepting the fact that this human body and mind are far from perfect. For one thing, we have kamma, a past that we carry around with us which is very difficult to shed.

For example, when we contemplate the precept about refraining from incorrect speech, we have the opportunity to learn to not create more kamma with our words, and to prevent it from being another source of harm and suffering for ourselves or for other beings. Right speech (\textit{samm\=a v\=ac\=a}) is one of the most difficult precepts because our words can reveal our thoughts and put us in a vulnerable situation. As long as we are silent, it's not so difficult. We can even seem quite wise, until we start talking. Those of you who have been on retreats may remember dreading having to relate verbally again with human beings. It's so nice, isn't it, just to be silent with each other; there are no quarrels, no conflicts. Silence is a great peacemaker! When we start talking, it's another ball game.

We can't really fool ourselves any longer. We tend to identify strongly with what we think, and so our speech, the direct expression of thoughts, also becomes a problem. But unless we learn to speak more skilfully, our words will continue to be quite hurtful to ourselves and to others. Actually speech itself is not so much the problem but the place it comes from. When there is mindfulness, there are no traces left behind. Sometimes we say something that is not very skilful, and afterwards, we think how we could have said it better. But if we speak mindfully, at that moment somehow the stain of that self-image that is so powerfully embedded in us is removed or, at least, diminished. As you follow this path of practice, discipline really makes sense. When you begin to get in touch with the raw energy of your being, and the raw energy of anger, greed, stupidity, envy, jealousy, blind desires, pride, conceit, you become very grateful to have something that can contain it all. Just look at the state of our planet, Earth: it is a great reflection on the harmful result of a lack of discipline and containment of our greed, hatred and delusion.

So we need to be very mindful and careful to be able to contain our energy within the framework of a moral discipline, because our mind's deepest  tendency is to forget itself. We forget ourselves and our lives' ultimate fulfilment, and instead fulfil ourselves with things that cannot truly satisfy or nourish our heart. This discipline also requires humility because, as long as we are immature and follow our impulses, we will feel repressed and inhibited by the discipline and, consequently, instead of being a source of freedom, we will feel trapped by it. We are very fortunate to have the chance to practise and realize that our actions, our speech or our desires are not ultimately what we are.

As our meditation deepens, the quality of impermanence of all things becomes clearer. We become more and more aware of the transient nature of our actions and speech, and our feelings related to these. We begin to get a sense for that which is always present in our experiences, yet is not touched by them. This quality of presence is always available and isn't really affected by our sensory interactions. When this quality of attention is cultivated and sustained we begin to relate more skilfully to our energy, to our sense contact and the sensory world. We discover that mindful attention is actually a form of protection. Without it, we're simply at the mercy of our thoughts or our desires, and get blinded by them. This refuge of awareness and the cultivation of restraint protect us from falling into painful hellish states of mind.

Another aspect of discipline is the wise attention and wise use of the material world. Our immediate contact with the physical world is through the body. When we learn how to take care of the physical world, we are looking after the roots of our lives. We do what is necessary to bring the body and mind into harmony. This is the natural outcome of restraint. Slowly, we become like a beautiful lotus flower that represents purity and grows out of the water while being nourished by its roots in the mud. You may have noticed how the Buddha is often depicted sitting on a lotus flower which symbolises the purity of the human heart. Unless we create that foundation of morality rooted in the world of our everyday life, we can't really rise up or grow like the lotus flower. We just wither.

In monastic life, the skilful use of the Four Requisites -- clothes, food, shelter and medicine -- is a daily reflection which is extremely useful because the mind is intent on forgetting, misinterpreting or taking things for granted. These four requisites are an essential part of our life. It is a duty for us monastics to care of our robes. We have to mend them, repair them, wash them and remind ourselves that we only have one set of them and that these robes have come to us through the generosity of others. The same goes for the food that we eat. We live on almsfood. Every day people offer us a meal because we are not allowed to store food for ourselves for the next day. So our daily reflection before the meal reminds us that we can't eat without thinking carefully about this gift of food. As alms mendicants, we also reflect on the place we live in. You may not like the wallpaper of your room, but the reflection on our shelter: `this room is only a roof over our head for one night' helps us to keep our physical needs in perspective. We consider also that without the offerings of these requisites we could not lead this life. This reflection nurtures a sense of gratitude in the heart.

Taking care of the physical world and what surrounds us is an essential part of the training of mind and body and of our practice of Dhamma. If we are not able to look after that which is immediate to us, how can we pretend to take care of the ultimate truths? If we don't learn to tidy our room every day, how can we deal with the complexities of our mind?

To reflect on simple things is very important, such as looking after our living place, and not misusing our material possessions. Naturally it is more difficult to do this when we have control over the material world and can use money to buy what we want, because we easily get careless, thinking: `Oh well I have lost this' or `I have broken that, never mind, I'll get another one.'

Another aspect of discipline and restraint is Right Livelihood. For a monk or a nun, there is a long list of things we should not get involved with, such as fortune-telling or participating in political activities, and so on. I can appreciate the value of this more and more as I see how, in some parts of the world where the Sa\.ngha has got involved in worldly issues, monks find themselves owning luxurious items or even becoming wealthy landlords. Right Livelihood is one aspect of the Noble Eightfold Path which covers a wide range of activities such as not deceiving, not persuading, hinting, belittling or bartering, and not involving ourselves in trades of weapons, living beings, meat, liquor, or poisons.

These guidelines call for a careful consideration of how we want to spend our life, and what kind of profession or situation we want to get involved in. The reflections on the precepts, the requisites, Right Livelihood and the discipline of our mind and body are the supportive conditions within which the ultimate discipline can manifest in our hearts. That ultimate discipline is our total dedication to the Truth, to the Dhamma, and the constant aspiration of our human heart to go beyond our self-centred lives. Sometimes we can't really say what it is, but through the practice of meditation we can be truly in touch with that reality, the Dhamma within ourselves. All spiritual paths and spiritual disciplines are here as supportive conditions and means for keeping alive this aspiration to realize Truth in our heart. That's really their aim.
