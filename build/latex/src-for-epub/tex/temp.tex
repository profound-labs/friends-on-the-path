
\textbf{Introduction}

We have put together this small collection of writings in response to an invitation to reprint some of the nuns' published teachings. There being no collection of teachings that were considered to be suitable for re-printing in its entirety, we have gathered material from various sources.

The Buddha used the simile of the great ocean which has one taste -- the taste of salt -- to point out that all of the Teachings (Dhamma) and Way of Practice (Vinaya) that he presented in his lifetime are concerned with just one thing: freedom from suffering. In the same way, although the teachings that appear in this collection were given over a significant time span and in different situations, the single intention is to point to causes of suffering and to offer suggestions as to how these causes can be eliminated.

The title of the book, `Friends on the Path', was chosen because it seems to be a theme in many of the teachings included here. We also recognize the central part that good friends (\textit{kaly\=a\d{n}amitta}) have played for each of us on our spiritual journey, and that the \textit{Dhamma-Vinaya}, the teachings and the training of the mind and body presented by the Buddha can also be seen as good friends and supports on the Path of liberation from suffering. We have been sharing this journey for more than thirty years and our continuing friendship through the ups and downs, disagreements and challenges of monastic life could also be seen as a testament to the efficacy of the Buddha's teachings and way of practice.
\bigskip

{\raggedleft\par
May all beings be free,\\
Ajahn Sundara and Ajahn Candasiri \par}

\textbf{Acknowledgements}

Ajahn Sundara's `Freedom within Restraint' and, `Simplicity' have been taken from the collection of nuns' teachings, `Freeing the Heart', which was first sponsored, edited and produced by Richard Smith in 2001. Similarly, `Why come to a Monastery' and, `A Question of Balance' from Ajahn Candasiri. `Fearless Presence' is a talk given by Ajahn Sundara at Seattle Insight in 2007 and `The Knowing Mind' is a reflection offered on a ten day retreat she taught in 2009 at Amaravati. Ajahn Candasiri's offering `Jesus through Buddhist eyes' is the transcript of a talk, given as part of the Radio 4 series, `Jesus through many eyes' which was produced by Norman Winter. It is reprinted from `Jesus in the World's Faiths: Leading thinkers from five religions reflect on his meaning', ed. Gregory A. Barker (New York: Orbis, 2008). `Me First' is from a talk by Ajahn Candasiri that appeared originally in `Awakening Presence' -- a collection of nuns' teachings that was sponsored and put together by Sumi Shin.

% In addition to those mentioned, the efforts of many people have contributed to the presentation of this collection of teachings. Adam Long has transcribed and edited several of the talks, and he and Talya Davies assisted greatly with proof-reading. Ajahn Amaro and Ajahn Jayanto have patiently responded to many queries about spelling and punctuation, while Ajahn Munindo and Samanera Gambhiro deserve special thanks for their encouragement, technical expertise and the great effort that has enabled the material to be prepared for publication.

% A final word of appreciation must go to the sponsor, who has chosen to remain anonymous.

\textbf{Ajahn Sundara} is French and was born in a liberal non-religious family. After studying dance, she worked for a few years as a dancer and teacher of contemporary dance. In her early thirties she encountered the Dhamma through Ajahn Sumedho's teachings and a 10 day retreat that he led in England. Her interest in Buddhist teachings grew, and in 1979 she joined monastic community of Chithurst Monastery where she was ordained as one of the first four women novices. In 1983 she was given the Going Forth as a \textit{s\={\i}ladhara} (ten precept nun) by Ajahn Sumedho. Since then she has participated in the establishment of the nuns' community, and for the last twenty years has taught and led meditation retreats in Europe and North America. She lives presently at Amaravati Buddhist Monastery.

\textbf{Ajahn Candasiri} is Scottish by birth and, like Ajahn Sundara, was one of the first nuns to be ordained by Ajahn Sumedho at Chithurst Monastery. Having been raised as a Christian, she continues to appreciate contact with contemplative Christians and with those of other faiths. Recognizing the immense benefit -- both for herself and others -- that can come about through a life of renunciation, she has actively participated in the evolution of the training and in providing opportunities for women to experience this form of practice. For most of her monastic life she has been resident at either Cittaviveka or Amaravati Monasteries.

% however, she hopes soon to establish a small monastery in Scotland for siladhara. She has taught retreats in the U.K. and abroad.

\textbf{Dedication}

Dedicated to the memory of our parents:

Jeanne \& Jean-Emile Reynaud

Kathleen \& Norman Cockburn

and to all the other wise teachers and friends who have shown us the Way.
