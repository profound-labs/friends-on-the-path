
% One Time \ldots{} 
% Ajahn Candasiri
% Extracted from a talk entitled, `Parting is such sweet sorrow', which was given at the end of a meditation retreat at Amaravati in 2000. 

One time, after being very ill I began to think of things I would regret not having done. One of the things that came to mind was that if, on reaching the end of my life, I felt that I hadn't really enjoyed it that would be sad. If I hadn't really chosen how I lived, being driven instead by the spectre of Duty hovering in the background, that would feel unfortunate. It became clear to me that as a disciple of the Buddha my `duty' is to be happy, to enjoy life; that being peaceful and at ease is, in fact, the fulfilment of his teachings -- which were offered for our welfare and happiness. So I began to contemplate what it is that brings happiness, a sense of welfare and ease. Two things came to mind: firstly, recalling of the goodness of my life and, secondly, taking good care of myself. 

I thought too that if I were to die with certain things uncompleted that that also would be a pity. For example, I realized that it was important to express gratitude to the people who have really helped me in my life. So I wrote to them, acknowledging what they had done and what it had meant to me. Then I remembered others I had hurt, so I wrote to them too: `Please forgive me for my insensitivity, the things I did that caused you unhappiness.' 

It's important to set things straight because although it is absolutely certain that we will all die one day, we can never know how long we've got before that will happen. Now I don't intend to make you feel panicky or frightened; I just want to encourage you to take care of things as you go along. Don't put it off. 

Consider: `What would I regret not doing?' And then take steps to do it, whenever it's convenient. It may happen that circumstances don't allow you to attend to it straight away, or maybe it'll never be possible, but at least you've taken it on board, you've taken responsibility for your aspiration. I was delighted, talking with a friend of the community whom I hadn't seen for quite some time. He told me he had just been to Antarctica. I thought, `Well, that's an unusual thing to do \ldots{}\thinspace' but it was something that he had always wanted to do; so off he went -- and had a wonderful time!

It's important to question, `Is this okay?', `Am I okay with this, have I really chosen to do this?' and if the answer is `no', to ask, `Is there something I can do to change it?' -- rather than accumulating a kind of subliminal grudge or feeling of bitterness -- that slightly sad, sour feeling that comes when we feel that we haven't really had any choice: `It's all because of \textit{them} that I had to do this!'

Sometimes we make assumptions about things we \textit{have} to do or what \textit{might} happen, without ever really questioning or considering: `Will everything fall apart if I don't do this?' Or we may find that if we set up the right conditions to have a frank discussion with the person who's been driving us nuts, rather than continually procrastinating, that it's all right -- in fact, quite a relief for all concerned!

I'm sure that everyone has a few conversations that are waiting to be had: conversations expressing appreciation, expressing tenderness or love. Perhaps we just assume that people know we love them, but have we actually told them? Or perhaps there are conversations that would be helpful: letting people know that things they are doing are really, really difficult for us. Can we attend to those? Or are we going to carry on allowing the irritation to gnaw away at us, draining away our sense of well-being year by year?

There is a phrase in the sharing of blessings chant: `May the forces of delusion not take hold, nor weaken my resolve.' It is through these `forces of delusion' that we allow ourselves to be undermined -- those subtle, unwholesome thoughts and moods that can creep into consciousness. We may barely be aware of them but then, as soon as we notice them, we push them down -- thinking that a good Buddhist must always be kind and loving; but why not try asking: `What's going on here?' Allow those miserable little beings lurking there in the shadows to show themselves. Try listening to them, try showing them a little concern. You'll be amazed at the increased level of joy and energy in your life, as you begin to allow these things into consciousness. They only gnaw away as long as we don't pay attention to them. That's all they're asking for; once they've been heard they can go.

So we need to take stock, to ask: `What's really important?', `What do I want to do with my life?', `Are there things I need to change?', `Are there ways I can live more happily?' Or, `Am I able, quite honestly and wholeheartedly, simply to rejoice in the way that my life is?' It may be that, for some of you, the answer is, `Yes. I'm content with my relationships, my home situation, my work \ldots{}\thinspace' I really hope that that is the case for you -- but if it's not, then it's good to consider ways to bring well-being and gladness into the heart.

If everything we do is from a sense of duty or a mild sense of bitterness or resentment, it's not going to bring a very happy result. The people we may be trying to help will pick up on it straight away; they'll feel that they're a burden, a nuisance. We don't want to make people feel like that, do we? It only really works when we've actually filled up our own heart with a sense of well-being -- so it naturally pours out; rather than just wringing the last ounce of kindness out of a poor, parched little heart. Sometimes it feels like that, doesn't it? When it's like that we really need to take stock and think, `What can I do to re-establish some sense of ease?'

Coming on retreat is something I would definitely recommend -- regularly drenching your whole being with goodness; contemplating the teachings, recollecting what is wholesome, what upholds a sense of dignity and self-respect -- then to consider how you can maintain this in your own life. For example, having a daily practice of recollection; meditating with others either in a monastery or in your own home; setting aside a day or half-day for a retreat in your own home; listening to tapes, reading Dhamma books -- these are all things that can nourish the heart.

It's good too to use the Five Precepts as a structure for practice in daily life. 

Generosity (\textit{d\=ana}) is another extremely beneficial practice, giving according to our means either materially, or time, energy or attention. For example, we can cultivate the practice of listening, really being present for another person -- this is an incredible act of generosity; taking ten minutes to really attend, to bear witness to their predicament -- knowing that we are doing this as much for our own well being as for theirs. Our time, our interest and practical help -- these count just as much as the more obvious practices of giving materially, in making the mind bright and joyous.
